\documentclass{article}
\title{Taller \#1. F\'isica Computacional / FISI 2025 \\Semestre 2013-II. \\ Profesor: Jaime E. Forero Romero}
\date{Agosto 8 2013}
\begin{document}
\maketitle

{\bf Los dos scripts de soluci\'on de esta tarea deben ser enviados por correo electr\'onico antes de las 5PM del jueves 22 de Agosto del 2013 a la direcci\'on del monitor del curso (Christian poveda) {\texttt{cn.poveda542@uniandes.edu.co}} con el subject \verb"RESPUESTA TALLER 2 FISICA COMPUTACIONAL"}.

\begin{enumerate}


\item Escriba un programa que diga si un n\'umero entero menor o igual a $10^6$ puede descomponerse como multiplicaci\'on de dos factores primos. Si es posible, que imprima los dos factores en pantalla. Si no es posible, que imprima un mensaje indicando que la separaci\'on en dos factores primos no se puede hacer.

\item Escriba un programa que cuente el numero de vocales (a,e,i,o,u) en un archivo ASCII arbitrario y que imprima en pantalla una l\'inea indicando la vocal y el n\'umero de veces que fu\'e encontrada.

\item Escriba un programa que calcule el centro de masa de una distribuci\'on de puntos en tres dimensiones. El archivo de entrada tiene un n\'umero indefinido de lineas y 3 columnas. Cada linea corresponde a un punto y las columnas a las posiciones $x$, $y$, $z$. 

\end{enumerate}

Cada punto tiene un valor de 

\end{document}
