\documentclass{article}
\textheight=25.5cm
\textwidth=16.0cm
\oddsidemargin=-0.5cm
\topmargin=-3.0cm
\usepackage[pdftex]{graphicx}
\usepackage[utf8]{inputenc}
\usepackage[spanish]{babel}
\title{Taller \#2 de M\'etodos Computacionales\\ FISI 2028, Semestre 2014 - 10}
\author{Profesor: Jaime Forero}
\date{Mi\'ercoles 5 de Febrero, 2014}
\begin{document}
\maketitle
\thispagestyle{empty}


{\bf Importante}
\begin{itemize}

\item Los siete archivos con el c\'odigo fuente que soluciona esta
  tarea deben subirse a trav\'es de sicuaplus antes de las 4PM del
  jueves 19 de Febrero como un \'unico archivo zip con el nombre
  \verb"NombreApellidos_hw2.zip", por ejemplo yo deber\'ia subir un
  archivo llamado \verb"JaimeForero_hw2.zip"

\item La nota m\'axima de este taller es de 100 puntos. Los puntos indicados
en cada literal solamente se otorgan si el programa compila y da los
resultados esperados seg\'un la descripci\'on de cada punto.
 

\item El archivo del genoma de la Vibrio cholerae se encuentra en este
  repositorio:

  \verb"https://github.com/forero/ComputationalMethodsData/tree/master/homework/hw_2"
\end{itemize}


\begin{enumerate}

\item 
\begin{itemize}
\item[a)] (10 puntos) Hay una araña en el borde de la rueda de un
  carro. La araña esta quieta mientras que el carro se mueve a una
  velocidad de $8$0 km/h. Las ruedas giran sobre el camino sin
  deslizamiento. Considere que el radio de la rueda es de 30
  cm. Haga un programa en C llamado \verb"mosca_quieta.c" que
  escriba dentro de un archivo llamado \verb"mosca_quieta.txt" las
  coordenadas $x$-$y$ (en metros) de la trayectoria que describe la mosca en un
  sistema de referencia en reposo. El archivo debe escribir en la
  primera columna la posici\'on en $x$ y en la segunda la posici\'on
  en $y$.

\item[b)] (10 puntos) Ahora la araña se mueve desde la parte exterior
  de la rueda hacia el centro con una velocidad de $30$ cm/s. Haga un
  programa en C llamado \verb"mosca_radial.c" que escriba dentro de un
  archivo llamado \verb"mosca_radial.txt" las coordenadas $x-y$ de la
  trayectoria de la mosca en un sistema de referencia en reposo. 
\end{itemize}

\item
\begin{itemize}
\item[a)] (10 puntos) Escriba un programa en C que genere un archivo
  con $n$ filas y $m$ columnas de n\'umeros aleatorios entre $0$ y
  $1$. El c\'odigo fuente debe estar en un archivo llamado
  \verb"gen_random.c". El programa debe poder ejecutarse como
  \verb"gen_random.x n m filename", donde \verb"filename" es un nombre
  arbitrario del archivo donde se van a escribir los datos. 

\item[b)] (10 puntos) 
Escriba un programa en C que lea un archivo de nombre arbitrario que
contiene $n$ filas y $m$ columnas de n\'umeros escritos en el mismo
formato que usa \verb"gen_random.c" para escribir los datos. El
c\'odigo fuente debe estar en un archivo llamado
\verb"max_random.c". El c\'odigo debe adem\'as imprimir en pantalla el
valor m\'aximo en cada una de las columnas. El programa debe poder
ejecutarse como \verb"max_random.x n m filename", donde
\verb"filename" es un nombre arbitrario del archivo que contiene los
datos. 

\item[c)] (25 puntos) Escriba un programa en C que lea un archivo de
  nombre arbitrario que contiene $n$ filas y $m$ columnas de n\'umeros
  escritos en el mismo formato que usa \verb"gen_random.c" para
  escribir los datos. El c\'odigo fuente debe estar en un archivo
  llamado \verb"sort_random.c". El c\'odigo debe adem\'as imprimir en
  pantalla las $m$ columnas con las $n$ filas reordenadas de tal
  manera que los valores de la primera columna est\'en ordenados de
  menor a mayor. El programa debe poder ejecutarse como 
  \verb"sort_random.x n m filename", donde \verb"filename" es un
  nombre arbitrario del archivo que contiene los datos. 

\end{itemize}

\item
\begin{itemize}
\item[a)] (10 puntos) El archivo \verb"Vibrio_cholerae.txt" contiene
  el genoma de la bacteria Vibrio cholerae. Escriba un programa en C
  llamado \verb"patron.c" que encuentre los {\bf dos} patrones de 5 bases
  consecutivas que m\'as se encuentren en las primeras $10^4$ bases
  del genoma de la Vibrio cholerae. 

\item[b)] (25 puntos) Escriba un programa en C llamado
  \verb"patron_nm.c" que encuentre los dos patrones de $n$ bases
  consecutivas que m\'as se encuentren en las primeras $m$ bases del
  genoma de la Vibrio cholerae. Este programa debe poder ejecutarse
  desde consola como \verb"patron_nm.x n m". El programa debe
  verificar que $n>0$, $m>0$, $n\leq m$ y que $m$ es menor que el
  n\'umero de bases en el genoma. Si alguna de esas condiciones no se
  cumple, el programa debe escribir un mensaje explicando el problema
  antes de parar su ejecuci\'on. 
\end{itemize}



\end{enumerate}

\end{document}
