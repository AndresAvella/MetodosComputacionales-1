\documentclass{article}
\title{Taller \#10. F\'isica Computacional / FISI 2025 \\Semestre
  2013-I. \\ Profesor: Jaime E. Forero Romero} 
\date{Abril 30 2013}
\begin{document}
\maketitle

{\bf Esta tarea debe resolverse por parejas (i.e. grupos de 2
  personas) y debe estar en un repositorio de la cuenta de github de
  uno de los miembros de cada equipo con un commit final hecho antes del
  medio d\'ia del viernes 10 de Mayo del 2013}  


El objetivo de este taller es escribir un integrador Runge-Kutta de
cuarto orden para estudiar la interacci\'on de dos galaxias de disco. Vamos a
seguir de cerca el paper cl\'asico de Toomre \& Toomre
\textit{Galactic Bridges and Tails},Astrophysical Journal, Vol. 178,
pp. 623-666 (1972), este paper esta en el repositorio como
\verb"homework/TT.pdf" 

Una galaxia ser\'a descrita como una masa central con 100
cuerpos en \'orbitas circulares que la rodean. Una imagen que puede
servir es la del sistema solar, pero donde solamente existen 5 orbitas
circulares posibles y en cada \'orbita se encuentran 12, 18, 24, 30 y 36
cuerpos, tal como se muestra en la Figura 1 del paper, en el panel
marcado con "$-1$". 

El movimiento de cada una de las part\'iculas que rodean a la masa
central est\'a determinado por la siguiente ecuaci\'on vectorial de
segundo orden:


\begin{equation}
\frac{d^2\vec{r}}{dt^2} = -\frac{GM}{r^2} \hat{r}
\end{equation}

donde $\vec{r}$  es un vector que va de la masa central a la
part\'icula, $\hat{r}$ es el vector unitario correspondiente,
$r=|\vec{r}|$, $M$ es la masa del cuerpo central y $G$ es la constante
de gravitaci\'on.  Una vez se conocen las condiciones iniciales de
cada una de las masas $\vec{r}_0$, $\vec{v}_0$ es posible conocer la
  posici\'on y la velocidad en eventos. Noten que estamos usando una
  aproximaci\'on donde la fuerza que siente cada part\'icula solamente
  se debe solamente a la masa central. Esto es equivalente a decir a
  que la masa de todas las part\'iculas es despreciable con respecto a
  la masa central. 


\begin{enumerate}
\item El primer punto de la tarea es escribir un programa que genere
  las condiciones  iniciales (posiciones y velocidades) para que las
  100 part\'iculas de una galaxia aislada   orbiten de manera estable
  en c\'irculos alrededor de la masa  central. La masa del cuerpo
  central debe ser de $10^{11}$ masas solares, el radio externo de
  $50$ kiloparsecs y cada una de las orbitas circulares debe estar
  equiespaciada a $10$ kiloparsecs.  
\item 
  Escriba el c\'odigo que evolucione la posici\'on y la velocidad de
  cada una de las part\'iculas durante 2 mil millones de
  a\~nos. 
\item 
  Tomando las condiciones iniciales anteriores, muestre que en efecto
  la configuraci\'on es estable. Prepare gr\'aficas de la posicion de
  las 100 part\'iculas en 5 momentos diferentes equiespaciados en los
  2mil millones de a\~nos de evoluci\'on del sistema.

\item 
  Interacci\'on con otra galaxia.
\end{enumerate}


Solamente se recibir\'an tareas que est\'en en un repositorio de
github. 

Enviar un email a {\tt  j.e.forero.romero} en {\tt gmail.com} con el
subject


\verb"RESPUESTA TALLER 10 FISICA COMPUTACIONAL". En el cuerpo del texto
debe ir la direcci\'on del repositorio donde est\'a la tarea.  


\end{document}
