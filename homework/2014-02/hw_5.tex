\documentclass{article}
\usepackage{hyperref}
\textheight=25.5cm
\textwidth=16.0cm
\oddsidemargin=-0.5cm
\topmargin=-3.0cm
\usepackage[pdftex]{graphicx}
\usepackage[utf8]{inputenc}
\usepackage[spanish]{babel}
\title{Taller \#5 de M\'etodos Computacionales\\ FISI 2028, Semestre 2014 - 20}
\author{Profesor: Jaime Forero}
\date{Mi\'ercoles 8 de Octubre, 2014}
\begin{document}
\maketitle
\thispagestyle{empty}


{\bf Importante}
\begin{itemize}

\item Todo el c\'odigo fuente y los datos se debe encontrar en un
  repositorio en github con un commit final hecho antes del medio
  d\'ia del martes 21 de Octubre. El nombre del repositorio debe ser
  \verb"Apellidos1_Apellidos2_hw5", por ejemplo si trabajo con
  Nicol\'as deber\'iamos crear un repositorio llamado
  \verb"Forero_Garavito_hw4".  

\item 
  La nota m\'axima de este taller es de 100 puntos. Se otorgan 1/3
  de los puntos si el c\'odigo fuente es razonable, 1/3 si se puede
  compilar/ejecutar y 1/3 si da los resultados correctos.  

\item
  El mi\'ercoles 15 de octubre tendremos una sesi\'on para presentar
  avances de diferentes grupos en cada una de las preguntas del
  taller. Hay un bono de 10 puntos para cada grupo que presente un
  avace significativo.  

\item 
  Todos los archivos para esta tarea se encuentran en
  \url{https://github.com/forero/ComputationalMethodsData/tree/master/homework/hw_5} 

\end{itemize}


\begin{enumerate}

\item {\bf Filtrando una imagen} (20 puntos) 

En clase trabajamos el filtrado de una se\~nal unidimensional
quit\'andole las frecuencias altas y dejando las frecuencias
bajas. Ahora vamos a intentar algo similar con una imagen, es decir con
una se\~nal bidimensional. Vamos a intentar dos cosas
diferentes: en un caso dejar las frecuencias altas y en otro caso
dejar las frecuencias bajas.

Escriba un programa en Python que haga el filtrado de la imagen
\verb"full_moon.jpg" de dos maneras. La primera que deje pasar las
frecuencias bajas; la segunda que deje pasar las frecuencias
altas. Cualitativamente hablando: ¿Qué hace cada filtro?.

En ambos casos implemente un filtro suave. Ver la siguiente
referencia: \url{http://paulbourke.net/miscellaneous/imagefilter/}. 


\item {\bf White Noise vs. Pink Noise} (40 puntos)
 
Escribir un programa en python que lea un archivo (.wav) de una
voz para convertirla en White Noise y Pink Noise. El programa debe
escribir la voz transformada en dos archivos .wav diferentes. El
archivo original de la voz sin transformar debe estar en el
repositorio de la tarea.


Ayuda:
\begin{itemize}
\item Para grabar archivos .wav en UNIX y en WINDOWS  se puede
  instalar SOX:
  \url{http://sox.sourceforge.net/}
  
\item 
  Para leer archivos de Sonido en python usar la libreria
  scikits.audiolab
 \url{http://cournape.github.io/audiolab/}
\end{itemize}


\item {\bf C\'irculos} (40 puntos)

En el archivo \verb"circulos.dat" se encuentran posiciones en un plano
$x-y$. Estos puntos corresponden a la superposici\'on de diferentes
c\'irculos m\'as un fondo de puntos distribuidos aleatoriamente a
partir de una distribuci\'on homog\'enea. 

Escriba un programa en python que encuentre el di\'ametro de estos
c\'irculos utilizando m\'etodos de transformada de Fourier.

Ayuda:
Funci\'on de autocorrelaci\'on \\ 
\url{http://mathworld.wolfram.com/Autocorrelation.html}\\
\url{http://mathworld.wolfram.com/Wiener-KhinchinTheorem.html}\\

\end{enumerate}


\end{document}
