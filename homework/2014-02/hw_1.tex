\documentclass{article}
\textheight=25.5cm
\textwidth=16.0cm
\oddsidemargin=-0.5cm
\topmargin=-3.0cm
\usepackage[pdftex]{graphicx}
\usepackage[utf8]{inputenc}
\usepackage[spanish]{babel}
\title{Taller \#1 de M\'etodos Computacionales\\ FISI 2028, Semestre 2014 - 20}
\author{Profesor: Jaime Forero}
\date{Viernes 1 de Agosto, 2014}
\begin{document}
\maketitle
\thispagestyle{empty}


{\bf Importante}
\begin{itemize}

\item Los tres scripts de soluci\'on de esta tarea deben subirse a
  trav\'es de sicuaplus antes de las medio d\'ia del viernes 15 de Agostocomo
  un \'unico archivo zip con el nombre
  \verb"NombreApellidos_hw1.zip", por ejemplo yo deber\'ia subir un
  archivo llamado \verb"JaimeForero_hw1.zip"

\item La nota m\'axima de este taller es de 100 puntos. Se otorgan 1/3 de los puntos si el script es razonable, 1/3 si se puede ejecutar y 1/3 si da los resultados correctos.

\item Todos los archivos se encuentran en este repositorio:

  \verb"https://github.com/forero/ComputationalMethodsData/"
\end{itemize}


\begin{enumerate}

\item El archivo \verb"hands_on/solar/monthrg.dat" contiene 5 columnas descritas por el archivo \verb"hands_on/solar/README". Escriba un script que imprima las siguientes tres cantidades.

\begin{itemize}
\item (10 puntos) El n\'umero de manchas solares promedio el mes de su nacimiento.
\item (10 puntos) C\'uantos  meses en entre 1900 y 1950 tuvieron m\'as de 30 manchas solares en promedio.
\item (10 puntos) El a\~no y el mes que m\'as manchas solares promedio ha tenido en toda la historia. 
\end{itemize}


\item
El archivo \verb"homework/hw_1/notas_fisicaII_201320.dat" contiene notas de
F\'isica II. Las primeras tres columnas son las notas de parciales, la
cuarta columna es la nota de la complementaria, la quinta columna es
la nota del final y la ultima columna es la nota definitiva. Las notas se encuentran sobre 100. Una nota aprobatoria corresponde a 60 o m\'as.

Escriba un script que responda a las siguientes 2 preguntas:
\begin{itemize}
\item (15 puntos) ¿Cu\'antos estudiantes perdieron exactamente un parcial y pasaron la materia?
\item (15 puntos) ¿Cu\'ales fueron el mejor y el peor promedio de parciales entre las personas que pasaron la materia?
\end{itemize}

\item
Un formato muy com\'un para archivos de datos es CSV (Comma Separated Value) donde las columnas se separan por comas. Los datos del archivo \verb"homework/hw_1/giro_2014.csv" se encuetran en este formato. 

Este archivo contiene información de todos los corredores del último Giro d'Italia. El tiempo que aparece en las columnas corresponde al tiempo acumulado al final de la etapa correspondiente.

Se puede utilizar \verb"awk" de la siguiente manera para extraer la columna 3 del archivo \verb"giro_2014.csv"
\begin{verbatim}
awk -F "\"*,\"*" '{print $3}' giro_2014.csv
\end{verbatim}

Escriba un script que responda a las siguientes preguntas:

\begin{itemize}
\item (10 puntos) ¿Cuántas nacionalidades diferentes participaban en el Giro?
\item (10 puntos) ¿Cuántos equipos diferentes participaron el Giro?
\item (20 puntos) ¿Quién era el l\'ider al final de cada una de las 21 etapas? 
\end{itemize}

\end{enumerate}

\end{document}
