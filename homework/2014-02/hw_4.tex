\documentclass{article}
\usepackage{hyperref}
\textheight=25.5cm
\textwidth=16.0cm
\oddsidemargin=-0.5cm
\topmargin=-3.0cm
\usepackage[pdftex]{graphicx}
\usepackage[utf8]{inputenc}
\usepackage[spanish]{babel}
\title{Taller \#4 de M\'etodos Computacionales\\ FISI 2028, Semestre 2014 - 20}
\author{Profesor: Jaime Forero}
\date{Miercoles 17 de Septiembre, 2014}
\begin{document}
\maketitle
\thispagestyle{empty}


{\bf Importante}
\begin{itemize}

\item Todo el c\'odigo fuente y los datos se debe encontrar en un
  repositorio en github con un commit final hecho antes del medio
  d\'ia del martes 7 de Octubre. El nombre del repositorio debe ser
  \verb"NombreApellidos_hw4", por ejemplo yo deber\'ia crear un
  repositorio llamado \verb"JaimeForero_hw4". Los datos se encuentran en el directorio \verb"homework/hw_4/Brahe-3141-f" del repositorio \url{https://github.com/forero/ComputationalMethodsData}. 

\item 
  La nota m\'axima de este taller es de 100 puntos. Se otorgan 1/3
  de los puntos si el c\'odigo fuente es razonable, 1/3 si se puede
  compilar/ejecutar y 1/3 si da los resultados correctos.  

\item
  Si se entrega la tarea antes del medio d\'ia del viernes 3 de
  Octubre los puntos se calificar\'an sobre 35-30-20-35, es decir la nota
  m\'axima posible es 120 en ese caso.

\end{itemize}


Es el a\~no 2150. Despu\'es de los descubrimientos de exoplanetas
similares a la Tierra en el 2020 y de haber resuelto en el 2080 el
problema de viaje inter-estelar en escalas de tiempo humanas, es una
pr\'actica com\'un para los estudiantes de la Universidad de los Andes
hacer salidas de campo a otros planetas.
 
En una de estas salidas de campo, el objetivo de los estudiantes de
las 300 secciones de Fisica I (la Universidad ahora cuenta con 500 mil
estudiantes) consiste en repartirse sobre la superficie del Planeta
Brahe-$3141$-f, similar a la Tierra en su masa y radio,
para hacer experimentos de movimiento parab\'olico y deducir el valor
de la gravedad en diferentes lugares del planeta.

Sus experimentos de alt\'isima precisi\'on (con errores en mediciones
de tiempos y posiciones despreciables) en c\'amaras gigantes de alto
vac\'io consisten en hacer tiros parab\'olicos y medir durante 4
segundos la trayectoria del proyecto.

Para evitar sesgos en las mediciones, cada uno de los tiros
parab\'olicos en cada una de las 1000 posiciones sobre Brahe-$3141$-f
tiene diferentes velocidades iniciales.

Hay 1000 archivos
diferentes con los datos de estos experimentos. Desafortunadamente
cerca de 100 archivos es\'tan corruptos y tienen datos con posiciones
que corresponden a ruido en la medici\'on.

Los archivos tienen nombres del tipo:
\begin{center}
\verb"experiment_theta_45.0_phi_45.0.dat"\\
\end{center}

Eso indica que ese es el experimento hecho a
$45$ grados medidos desde el polo norte y a $45$ grados desde el
meridiano principal de Brahe-$3141-f$.


El objetivo de la tarea es escribir un c\'odigo en Python (o en un
notebook de IPython) que haga las siguientes tareas:
\begin{enumerate}

\item (30 puntos)
Por cada uno de los archivos de datos devuelva los par\'ametros
$g_0,v_{0y}, y_{0}$, que corresponden a la aceleraci\'on de la gravedad
en ese sitio, la velocidad inicial y la posici\'on inicial. {\bf Esto se debe
hacer usando la versi\'on matricial de m\'inimos cuadrados. La rutina
que hace la diagonalizaci\'on, descomposici\'on LU o descomposici\'on de Cholesky deben escribirla
ustedes mismos.} Los par\'ametros deben quedar escritos en un archivo
de 5 columnas, donde las primeras dos columnas corresponden a los
valores de $\theta$ y $\phi$ y las otras tres columnas corresponden a
los par\'ametros del movimiento parab\'olico. 


\item (25 puntos)
Prepara una gr\'afica de los valores de la gravedad como funci\'on del
\'angulo polar $\theta$, descartando los resultados de archivos con
datos corruptos.

\item (15 puntos)
El programa debe preparar una lista de las variaciones de la gravedad
parametrizada por 

\begin{displaymath}
F=1-\frac{g_{0}}{\langle g_{0}\rangle},
\end{displaymath}

en funci\'on del \'angulo polar $\theta$ y mostrarlas en una
gr\'afica, donde $\langle g_{0} \rangle$ es el valor medio de todas las mediciones. 

\item (30 puntos)
F corresponde a las fluctuaciones de la gravedad con respecto a su
valor medio.

Hay dos hip\'otesis para la dependencia de $F$ con respecto al
\'angulo. La primera hip\'otesis dice
\begin{equation}
F = a_{0}\cos(2\theta) + a_{1},
\end{equation}
donde $a_{0}$ y $a_{1}$ son constantes.

La segunda hip\'otesis dice

\begin{equation}
F = b_{0} + b_{1}\theta + b_{2}\theta^2, 
\end{equation}

donde $b_{0}$, $b_{1}$ y $b_{2}$ son constantes a determinar.

Encuentre los valores de las constantes usando el mismo m\'etodo de
m\'inimos cuadrados y justifique cu\'al de los dos modelos es mejor
para describir los datos. 


\end{enumerate}


\vspace{1cm}

Para ver una aplicaci\'on real de cartograf\'ia de la Luna a
partir de mediciones de precision de fluctuaciones del campo
gravitacional, pueden ir aqu\'i:

\url{http://www.nasa.gov/mission_pages/grail/news/grail20121205.html}

\end{document}
