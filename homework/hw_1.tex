\documentclass{article}
\title{Taller \#1. F\'isica Computacional / FISI 2025 \\Semestre 2013-II. \\ Profesor: Jaime E. Forero Romero}
\date{Agosto 1 2013}
\begin{document}
\maketitle

{\bf Los dos scripts de soluci\'on de esta tarea deben ser enviados por correo electr\'onico antes del medio d\'ia del jueves 8 de Agosto del 2013 a la direcci\'on del monitor del curso (Christian poveda) {\texttt{cn.poveda542@uniandes.edu.co}} con el subject \verb"RESPUESTA TALLER 1 FISICA COMPUTACIONAL"}.

\begin{enumerate}

\item
El objetivo de este punto crear un script que tenga el nombre \verb"NombreApellido_saber.sh" donde donde \verb"NombreApellido" lo deben reemplazar por su nombre y apellido. Por ejemplo, yo crear\'ia el archivo \verb"JaimeForero_saber.sh". El script debe ejecutar las siguientes acciones:

\begin{enumerate}
\item 
Traer el archivo  \verb"http://www.finiterank.com/saber/2011.csv"
\item
Traer el archivo \verb"https://raw.github.com/forero/"\newline
\verb"ComputationalPhysicsUniandes/master/hands_on/unix/columnas_2011.csv.txt"
\item

  Utilizando el siguiente tipo de comando para seleccionar una columna del archivo \verb"2011.csv"
\begin{verbatim}
awk -F "\"*,\"*" '{print $3}' 2011.csv
\end{verbatim}
donde en el caso anterior \verb"$3" corresponde a la columna 3 del archivo seleccionado, imprimir en pantalla los siguientes n\'umeros:
\begin{itemize}
\item Cu\'antos colegios p\'ublicos hay.
\item Cu\'antos colegios privados hay.
\item Cu\'antos colegios privados de calendario A hay.
\item Cu\'antos colegios privados de calendario B hay.
\item Cu\'antos colegios publicos de calendario B hay.
\end{itemize}


\end{enumerate}
\item
El objetivo de este punto crear un script que tenga el nombre \verb"NombreApellido_contar.sh" donde donde \verb"NombreApellido" lo deben reemplazar por su nombre y apellido. Por ejemplo, yo crear\'ia el archivo \verb"JaimeForero_contar.sh". El script debe ejecutar las siguientes acciones:
\begin{enumerate}
\item Traer el archivo \verb"https://raw.github.com/forero/"\newline \verb"ComputationalPhysicsUniandes/master/hands_on/unix/random_3D.dat"
\item Imprimir en pantalla el n\'umero de veces que ocurren las cifras {\texttt{ 0 1 2 3 4 5 6 7 8 9}}
\end{enumerate}
\end{enumerate}

\end{document}
