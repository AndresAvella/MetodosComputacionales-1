\documentclass{article}
\title{Taller \#1. F\'isica Computacional / FISI 2025 \\Semestre 2013-I. \\ Profesor: Jaime E. Forero Romero}
\date{Enero 31 2013}
\begin{document}
\maketitle

{\bf Esta tarea debe ser enviada por correo electr\'onico antes del medio d\'ia del martes 5 de Febrero del 2013 a la direcci\'on {\tt j.e.forero.romero} en {\tt gmail} con el subject \verb"RESPUESTA TALLER 1 FISICA COMPUTACIONAL"}.

\begin{enumerate}
\item
Crear un archivo que tenga el siguiente nombre:\\
\verb"contacto_de_NombreApellido.txt"\\
donde \verb"NombreApellido" lo deben reemplazar por su nombre y apellido. Por ejemplo, yo crear\'ia el archivo \verb"contacto_de_JaimeForero.txt".

\item
Este archivo debe contener cuatro l\'ineas: en la primera el nombre completo de un conocido de ustedes, en la segunda la fecha de la \'ultima vez que tuvieron contacto con el/ella, en la tercera l\'inea la ciudad y pa\'is donde vive esta persona y en la cuarta l\'inea un mensaje breve de 140 caracteres como m\'aximo.
El formato \underline{debe} ser el siguiente

\begin{verbatim}
Lucia Ayala
3 Enero 2013
Berkeley, USA
Sin importar lo que pase, siempre te estare esperando.
\end{verbatim}

\item
Aparte deben escribir un programa en C que se llame\\ \verb"carta_de_NombreApellido.c"\\ donde \verb"NombreApellido" se reemplaza por su nombre y apellidos. 

El c\'odigo debe poder compilarse asi:
\begin{verbatim}
cc carta_de_NombreApellido.c
\end{verbatim}


El programa debe tomar cuatro argumentos:\\
 el archivo \verb"contacto_de_NombreApellido.txt", un dia, un mes y un a\~no y debe poder ejecutarse as\'i:
\begin{verbatim}
./a.out  contacto_de_NombreApellido.txt 31 Enero 2013
\end{verbatim}

Al ejecutarse el programa debe producir el siguiente resultado

\begin{verbatim}
Bogota [Fecha]
Hola [Nombre y Apellidos del conocido]'
Hace unos [X] dias que no te escribo, por eso quisiera aprovechar 
este momento para decirte algo que siempre  pense en decir: 

[Mensaje dentro del archivo de texto]

Espero que nos podamos ver pronto en [ciudad], siempre he 
querido visitar [pais].

Saludos,
[Nombre]
\end{verbatim}

\item
Enviar por correo el codigo fuente de C y el archivo \verb".txt". Siguiendo el ejemplo, yo enviar\'ia dos archivos:\\ \verb"carta_de_JaimeForero.c" \\ \verb"contacto_de_JaimeForero.txt"

\item
La calificaci\'on tendr\'a tres partes consecutivas. 
\begin{itemize}
\item El c\'odigo fuente debe compilar (30\%).
\item El c\'odigo debe funcionar de manera adecuada con el archivo del amigo que cada uno env\'ia y una fecha arbitraria de entrada (30\%).
\item El c\'odigo debe funcionar de manera adecuada con un archivo de texto que yo produzco y una fecha arbitraria de entrada (40\%).
\end{itemize}

Nota: No tomen en cuenta los a\~nos bisiestos ni los cambios hist\'oricos de calendario. Para este taller todos los a\~nos han tenido y tendr\'an 365 d\'ias.
\end{enumerate}
\end{document}
