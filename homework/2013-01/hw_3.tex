\documentclass{article}
\title{Taller \#3. F\'isica Computacional / FISI 2025 \\Semestre 2013-I. \\ Profesor: Jaime E. Forero Romero}
\date{Febrero 19 2013}
\begin{document}
\maketitle

{\bf Esta tarea debe estar en un repositorio de la cuenta de github de
  cada uno con un commit final hecho antes del medio d\'ia del jueves
  28 de Febrero del 2013} 

\begin{enumerate}
\item
Dentro del repositorio del curso vayan al directorio
\verb"hands_on/lin_algebra". 

\item
Ahi encontrar\'an el archivo \verb"movimiento.dat" que tiene 2
columnas. La primera representa una variable temporal y la segunda una
posici\'on. Corresponden a las mediciones de la posici\'on de un
cuerpo en un campo gravitacional. 

Siguiendo la parametrizaci\'on que se da en las notas de clase,
escriba un programa en C que calcule el valor \'optimo de esos
par\'ametros. Los c\'alculos deben utilizar expl\'icitamente la
formulaci\'on matricial del problema inverso.

\item
El output del c\'odigo debe ser un archivo de texto con tres floats
escritos en la misma fila, correspondientes a los par\'ametros $m_1$,
$m_2$ y $m_3$. El archivo de salida se debe llamar \verb"parametros_movimiento.dat".  

\item

En el mismo directorio encontrar\'an el archivo \verb"3D_data.dat" que
tiene 3 columnas. Cada una de las filas
corresponde a una medici\'on de un elemento descrito por tres
n\'umeros $x_{1}, x_{2}, x_{3}$. Explorando la distribuci\'on de estos
n\'umeros en el espacio de las variables, es claro que en realidad
estos datos viven en un espacio bidimensional. El objetivo es escribir
un c\'odigo en C que realiza una an\'alisis del tipo Principal
Component Analysis para poder descubrir en que plano del espacio
$x_{1}, x_{2}, x_3$ viven las variables \verb"3D_data.dat". 

\item 
El output del c\'odigo debe ser un archivo de texto con los tres
autovectores ordenados por orden decreciente del autovalores. Cada
autovector debe estar escrito en una fila por separado, cada
columna corresponde entonces a las componentes en
$x_{1}, x_{2}, x_{3}$ . Estos vectores deben tener norma unidad. El archivo de
salida, despu\'es de ejecutar el c\'odigo debe ser
\verb"autovectores_3D_data.dat". El archivo debe poder compilarse con
las librerias \verb"-lm -lgsl -lgslcblas".

\item
Los dos c\'odigos deben estar en un repositorio de GitHub dentro de
un directorio que se llame \verb"lin_algebra".

\item
Enviar un email al monitor del curso Daniel Felipe Duarte {\tt
  df.duarte578} en {\tt uniandes.edu.co} con el subject
\verb"RESPUESTA TALLER 3 FISICA COMPUTACIONAL". En el cuerpo del texto
debe ir la direcci\'on del repositorio donde est\'a la tarea. 

\item
La calificaci\'on se har\'a en tres partes.
\begin{itemize}
\item El c\'odigo fuente y los libros est\'an en un repositorio de
  github (20\%). 
\item El c\'odigo de m\'inimos cuadrados recupera adecuadamente los
  par\'ametros que describen el movimiento dado por
  \verb"movimiento.dat"(20\%).
\item El c\'odigo de m\'inimos cuadrados recupera adecuadamente los
  par\'ametros que describen el movimiento de un conjunto de datos
  diferente (30\%).
\item El c\'odigo de PCA recupera adecuadamente los autovectores
  autovectores que describen los datos \verb"3D_data.dat"(20\%) 
\item El c\'odigo de PCA recupera adecuadamente los autovectores que
  describen un conjunto de datos diferente (tambien en 3 dimensiones) (30\%).
\end{itemize}
\end{enumerate}
\end{document}
