\documentclass{article}
\title{Taller \#10. F\'isica Computacional / FISI 2025 \\Semestre
  2013-I. \\ Profesor: Jaime E. Forero Romero} 
\date{Mayo 7 2013}
\begin{document}
\maketitle

{\bf Esta tarea debe resolverse por parejas (i.e. grupos de 2
  personas) y debe estar en un repositorio de la cuenta de github de
  uno de los miembros de cada equipo con un commit final hecho antes del
  medio d\'ia del viernes 10 de Mayo del 2013}  


El objetivo de este taller es escribir un integrador Runge-Kutta de
cuarto orden para estudiar la interacci\'on de dos galaxias de disco. Vamos a
seguir el ejemplo  del paper cl\'asico de Toomre \& Toomre
\textit{Galactic Bridges and Tails} en el Astrophysical Journal, Vol. 178,
pp. 623-666 (1972), el cual se encuentra repositorio como
\verb"homework/TT.pdf" 

Una galaxia ser\'a descrita como una masa central con 100
cuerpos en \'orbitas circulares que la rodean. Una imagen que puede
servir es la del sistema solar, pero donde solamente existen 5 orbitas
circulares posibles y en cada \'orbita se encuentran 12, 18, 24, 30 y 36
cuerpos, tal como se muestra en la Figura 1 del paper, en el panel
marcado con "$-1$". 

El movimiento {\bf de cada una} de las part\'iculas que rodean a la masa
central est\'a determinado por la siguiente ecuaci\'on diferencial
ordinaria vectorial de segundo orden: 
 

\begin{equation}
\frac{d^2\vec{r}}{dt^2} = -\frac{GM}{r^2} \hat{r}
\end{equation}

donde $\vec{r}$  es un vector que va de la masa central a la
part\'icula, $\hat{r}$ es el vector unitario correspondiente,
$r=|\vec{r}|$, $M$ es la masa del cuerpo central y $G$ es la constante
de gravitaci\'on.  Una vez se conocen las posiciones y velocidades
iniciales de
cada una de las masas $\vec{r}_0$, $\vec{v}_0$ es posible conocer la
  posici\'on y la velocidad en momentos siguientes. Noten que estamos usando una
  aproximaci\'on donde la fuerza que siente cada part\'icula solamente
  se debe solamente a la masa central. Esto es equivalente a decir a
  que la masa de todas las part\'iculas es despreciable con respecto a
  la masa central. As\'i mismo la masa central no siente ninguna
  fuerza apreciable por parte de las part\'iculas.

Los siguientes puntos deben ser desarrollados en Python.

\begin{enumerate}
\item El primer punto de la tarea consiste escribir un programa que genere
  las condiciones  iniciales (posiciones y velocidades) para que las
  100 part\'iculas de una galaxia de disco orbiten de manera estable
  en c\'irculos alrededor de la masa  central. La masa del cuerpo
  central debe ser de $10^{12}$ masas solares, el radio externo de
  $50$ kiloparsecs y cada una de las orbitas circulares debe estar
  equiespaciada a $10$ kiloparsecs (1 parsec son
  $3.0\times10^{16}$metros).  
Pista: ¿Cu\'al es la aceleraci\'on centr\'ipeta de una part\'icula en una
\'orbita de radio $r$?
\item 
  Escriba el c\'odigo que evolucione la posici\'on y la velocidad de
  cada una de las part\'iculas durante 2 mil millones de
  a\~nos.  Se debe utilizar un m\'etodo de Runge-Kutta de cuarto orden
  para integrar la ecuaci\'on (1) para cada part\'icula del disco. 
\item 
  Tomando las condiciones iniciales anteriores, muestre que en efecto
  la configuraci\'on es estable. Es decir, las part\'iculas que
  representan el disco de la galaxia siguen en su \'orbita circular
  despu\'es de los 2 mil millones de a\~nos. Prepare gr\'aficas de la posicion de
  las 100 part\'iculas en 5 momentos diferentes equiespaciados en los
  2mil millones de a\~nos de evoluci\'on del sistema.
\item 
  El punto final incluye una segunda galaxia id\'entica a la
  primera. Si se considera que la galaxia anterior tiene un centro de
  masa en la posici\'on $(0\hat{i}+0\hat{j})$ kpc una velocidad de centro de
  masa nula, vamos a considerar ahora que la masa central de la segunda galaxia
  tiene una posicion $(150\hat{i}+200\hat{j})$ kpc y el centro de masa
  tiene una velocidad inicial $-100\hat{j}$ km/s. La
  direcci\'on $\hat{z}$ es perpendicular al plano del disco. Ahora
  evolucione estas nuevas condiciones iniciales por 2 mil millones de
  a\~nos para ver la interacci\'on de las dos galaxias.  Prepare gr\'aficas de la posicion de
  las part\'iculas de las dos galaxies en 5 momentos diferentes equiespaciados en los  2mil millones de a\~nos de evoluci\'on del sistema.

  Note que en esta configuraci\'on las masas centrales sienten su
  influencia mutua y su \'orbita tambi\'en debe ser calculada.

\end{enumerate}



En la calificaci\'on se dar\'a un 25\% a cada uno de los puntos del 1
al 4. Solamente se recibir\'an tareas que est\'en en un repositorio de
github. 

Enviar un email a {\tt  j.e.forero.romero} en {\tt gmail.com} con el
subject


\verb"RESPUESTA TALLER 10 FISICA COMPUTACIONAL". En el cuerpo del texto
debe ir la direcci\'on del repositorio donde est\'a la tarea.  


\end{document}
