\documentclass{article}
\textheight=25.5cm
\textwidth=16.0cm
\oddsidemargin=-0.5cm
\topmargin=-3.0cm
\usepackage[pdftex]{graphicx}
\usepackage[utf8]{inputenc}
\usepackage[spanish]{babel}
\title{Taller \#5 de M\'etodos Computacionales\\ FISI 2028, Semestre 2014 - 10}
\author{Profesor: Jaime Forero}
\date{Mi\'ercoles 12 de Marzo, 2014}
\begin{document}
\maketitle
\thispagestyle{empty}


{\bf Importante}
\begin{itemize}

\item Todos los programas que solucionan esta tarea deben encontrarse
  en un repositorio en github con un commit final hecho antes del
  medio d\'ia del viernes 28 de Marzo. Cada literal debe resolverse
  con un noteboook de ipython por separado. 

\item La nota m\'axima de este taller es de 100 puntos. Los puntos indicados
  en cada literal solamente se otorgan si el da los resultados
  esperados seg\'un la descripci\'on de cada punto. 
 

 \item Los datos de esta tarea se encuentran en el repositorio


\verb"https://github.com/forero/ComputationalMethodsData" 

en la carpeta \verb"homework/hw_5".

\end{itemize}

\begin{enumerate}
\item
{\bf Per\'iodo del ciclo solar} (20 puntos)
En \verb"sparse_sample_monthrg.dat" se encuentran datos del n\'umero
de manchas solares en funci\'on del tiempo. La primera columna
corresponde al a\~no, la segunda al mes, la tercera al n\'umero de
d\'ias de datos tomados y la cuarta al promedio de manchas. La
particularidad de estos datos es que {\bf no est\'an espaciados
  homog\'eneamente en el tiempo.} 

Escriba un programa en Python que estime el ciclo solar en a\~nos a
partir de an\'alisis de Fourier.


\item
{\bf Estrellas Variables RR-Lyrae}
Datos de la intensidad de una estrella variable RR-Lyrae se encuentran
en \verb"RR_Lyrae_template.dat". 

\begin{itemize}
\item[a)] (10 puntos) Escriba un programa en Python que calcule la
  misma curva de intensidad cuando se toman en cuenta $N$ ($1<N<11$)
  componentes de Fourier. 
\item[b)] (5 puntos) Prepare gr\'aficas de la curva reconstru\'ida con $N$
componentes.  
\item[c)] (5 puntos)
Prepara una gr\'afica de $\chi^2$ en funci\'on del n\'umero de
componentes $N$ tomadas en cuenta al momento de hacer la
reconstrucci\'on.   
\end{itemize}

\item

{\bf C\'irculos} (30 puntos)
En el archivo \verb"BAO.dat" se encuentran posiciones en un plano
$x-y$. Estos puntos corresponden a la superposici\'on de diferentes
c\'irculos m\'as un fondo de puntos distribuidos aleatoriamente a
partir de una distribuci\'on homog\'enea. 

Escriba un programa en python que encuentre el di\'ametro de estos
c\'irculos. 

Ayuda:
Funci\'on de autocorrelaci\'on \\ 
\verb"http://mathworld.wolfram.com/Autocorrelation.html"\\
\verb"http://mathworld.wolfram.com/Wiener-KhinchinTheorem.html"

\item {\bf Filtro pasa bandas}(30 puntos) 
Escribir un programa en python que lea un archivo (.wav) y
  calcule la transformada de Fourier, haga cero las amplitudes por fuera de una
banda de frecuencias, y luego calcula la tranformada inversa para escribir
un archivo (.wav) con el resultado.

Ayuda:
\begin{itemize}
\item Para grabar archivos .wav en UNIX y en WINDOWS  se puede
  instalar SOX:
  \verb"http://sox.sourceforge.net/"
  
\item 
  Para leer archivos de Sonido en python usar la libreria
  scikits.audiolab
 \verb"http://cournape.github.io/audiolab/"
\end{itemize}

\end{enumerate}

\end{document}
