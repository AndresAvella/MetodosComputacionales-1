\documentclass{article}
\usepackage{verbatim}

\begin{document}
\section[2]{Basics of C programming}
This chapter will serve two purposes. First it will show the basic aspects of using a low level language (like FORTRAN or C++) and will introduce the basic features of the C language that will be useful for the rest of the semester.

There are many good manuals and texts to learn C. The classic book is 
\begin{itemize}
\item The C Programming Language, Kernighan and Ritchie, Second Edition, Prentice Hall, 1988.
\end{itemize}

The book is so popular that you should be able to find at least one pdf copy by googling \verb"kernighan and ritchie pdf". It is important to say that Ritchie invented C and Kernighan wrote the first C tutorial ever.

You should also try to find a good, modern tutorial on the web. In this notes I will follow closely the structure in Kernighan and Ritchie.

Why C and not FORTRAN?
C is widely used outside academia. It is good skill to have if you are interested in having a job in interesting places outside universities and scientific laboratories. 
The C compilers are now an standard and are commonly included in any UNIX based operative system.

Why FORTRAN and not C?

The reality is that you need to learn at least one low level language in order to be productive and useful in academia. Actually, you should learn both.


\subsection{Write, Compile, Execute}
Using a low level programming language implies that there is text that must be written into a file. This file will receive the generic name of source code. This file can be editted using any plain text editor. Try to learn EMACS or VIM. In this course we will use EMACS.

\subsubsection{Write}
Let us first create a working directory called \verb"practice/C/" inside the working directory you have for this class.
\begin{verbatim}
$ mkdir -p hands_on/C/
$ cd hands_on/C/
\end{verbatim}

Now open in emacs a file called \verb"hello.c"
\begin{verbatim}
$ emacs hello.c &
\end{verbatim}

There you can type the following piece of source code

\verbatiminput{../hands_on/C/hello.c}
\subsection{A more useful example deconstructed}

% Making a table of numbers, control flow, arithmetic

\subsection{Variable types}

\subsection{Arrays}

\subsection{Memory allocation and pointers}

\subsection{String Arrays}

\subsection{If, do-while}

\subsection{Your own functions}

\subsection{Input/Output from files}

\subsection{Dealing with multiple source files}

\subsection{Command line input}

\subsection{The same things}

\end{document}
