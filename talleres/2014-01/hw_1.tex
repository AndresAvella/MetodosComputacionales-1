\documentclass{article}
\textheight=25.5cm
\textwidth=16.0cm
\oddsidemargin=-0.5cm
\topmargin=-3.0cm
\usepackage[pdftex]{graphicx}
\usepackage[utf8]{inputenc}
\usepackage[spanish]{babel}
\title{Taller \#1 de M\'etodos Computacionales\\ FISI 2028, Semestre 2014 - 10}
\author{Profesor: Jaime Forero}
\date{Mi\'ercoles 22 de Enero, 2014}
\begin{document}
\maketitle
\thispagestyle{empty}


{\bf Importante}
\begin{itemize}

\item Los tres scripts de soluci\'on de esta tarea deben subirse a
  trav\'es de sicuaplus antes de las 4PM del jueves 30 de Enero como
  un \'unico archivo zip con el nombre
  \verb"NombreApellidos_hw1.zip", por ejemplo yo deber\'ia subir un
  archivo llamado \verb"JaimeForero_hw1.zip"

\item La nota m\'axima de este taller es de 100 puntos. Los puntos indicados
en cada literal solamente se otorgan si el script se puede ejecutar y
da los resultados correctos.  

\item Todos los archivos se encuentran en este repositorio:

  \verb"https://github.com/forero/ComputationalMethodsData/tree/master/homework/hw_1"
\end{itemize}


\begin{enumerate}

\item El archivo \verb"hamlet.txt" contiene a Hamlet. Escriba un
  script que calcule lo siguiente
\begin{itemize}
\item (10 puntos) ¿Cu\'antas l\'ineas no contienen la palabra \verb"the"?
\item (20 puntos) ¿Cu\'antas veces aparece la vocal que m\'as veces est\'a en el
  texto? ¿Cuál es esta vocal? 
\end{itemize}

\item
El archivo \verb"Pi_2500000.txt" contiene las primeras 2500000 cifras
decimales de $\pi$.  Haga un script que 
\begin{itemize}
\item Encuentre cuantas veces aparecen los literales \verb"0",
  \verb"00", \verb"000", \verb"0000", \verb"00000", \verb"000000".
\item Encuentre cuantas veces aparecen los literales \verb"N",
  \verb"N0", \verb"N00", \verb"N000", \verb"N0000", \verb"N00000",
  donde \verb"N" es un n\'umero que va de 1 hasta 9.
\item (20 puntos) Guarde los items anteriores en un archivo que tiene
  10 filas y 6 columnas. Cada columna corresponde a un valor de $N$ y
  el rango en la fila corresponde a las ocurrencias de cada uno de los
  literales para ese \verb"N".
\item ¿Notan alg\'un patr\'on en estos resultados? ¿Podría explicarlo?
\end{itemize}


\item
El archivo \verb"notas_fisicaII_201320.dat" contiene notas de
Fisica II. Las primeras tres columnas son las notas de parciales, la
cuarta columna es la nota de la complementaria, la quinta columna es
la nota del final y la ultima columna es la nota definitiva. 

Escriba un script que utilice \verb"awk" para responder a las
siguientes preguntas: 
\begin{itemize}
\item (10 puntos) ¿Cu\'antos estudiantes pasaron el primer parcial?
\item (10 puntos) ¿Cu\'antos estudiantes pasaron el primer parcial,
  perdieron el final, y pasaron la materia?
\item (15 puntos) ¿Cu\'antos estudiantes perdieron al menos un parcial
  y pasaron la materia?
\item (15 puntos)¿Cu\'antos estudiantes pasaron solamente dos
  parciales y pasaron la materia? 
\end{itemize}


\end{enumerate}

\end{document}
