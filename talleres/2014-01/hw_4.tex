\documentclass{article}
\textheight=25.5cm
\textwidth=16.0cm
\oddsidemargin=-0.5cm
\topmargin=-3.0cm
\usepackage[pdftex]{graphicx}
\usepackage[utf8]{inputenc}
\usepackage[spanish]{babel}
\title{Taller \#4 de M\'etodos Computacionales\\ FISI 2028, Semestre 2014 - 10}
\author{Profesor: Jaime Forero}
\date{Mi\'ercoles 26 de Febrero, 2014}
\begin{document}
\maketitle
\thispagestyle{empty}


{\bf Importante}
\begin{itemize}

\item Todos los programas que solucionan esta tarea deben encontrarse
  en un repositorio en github con un commit final hecho antes del
  medio d\'ia del viernes 14 de Marzo. Cada literal debe resolverse
  con un noteboook de ipython por separado.

\item La nota m\'axima de este taller es de 140 puntos. Los puntos indicados
  en cada literal solamente se otorgan si el da los resultados
  esperados seg\'un la descripci\'on de cada punto. 
 
\end{itemize}

\begin{enumerate}
\item
{\bf Homicidios}\\

En el siguiente repositorio:

\verb"https://github.com/finiterank/homicidios/"

se encuentra el archivo \verb"data/homicidios.1990.a.2013.csv" que contiene
los datos de homicidios en todos los municipios de Colombia entre 1990
y 2013.

\begin{itemize}
\item[a)] (25 puntos) Realice un descomposici\'on en PCA de todos los
  historiales de homicidios expresados en la tasa de homicidios por
  cada 100 mil habitantes. ¿Es posible reducir la dimensionalidad de
  los datos?  

\item[b)] (25 puntos) Exprese ahora cada una de las historias de tasa de
  homicidios como la lista de coeficientes que son necesarios
  para expresarlas como combinación lineal de los autovectores
  encontrados en el ítem anterior. Haga un análisis de K-clustering
  sobre estos datos para $K=2,3,4,5$. ¿Es posible decir que los datos
  se separan en clusters? 
\end{itemize}




\item 
{\bf Pobreza}\\
En el siguiente repositorio:

\verb"https://github.com/finiterank/mapa_pobreza_colombia"

se encuentra el archivo \verb"data/pobreza.csv" con datos
multidimensionales de pobreza que incluyen acceso a salud, calidad
escolar, empleo informal, etc.

\begin{itemize}
  \item[a)] (25 puntos) Realice un estudio PCA sobre las variables
    \emph{Analfabetismo},
    \emph{Inasistencia Escolar}, \emph{Alta dependencia econ\'omica},
    \emph{Sin Fuente Agua Mejorada}. ¿Es posible reducir la
    dimensionalidad de los datos?
  \item[b)] (25 puntos) ¿Existen clusters en el espacio de esas mismas
    variables? 
\end{itemize}


\item 
{\bf Saber, Pobreza y Homicidios}\\
(20 puntos)

En el siguiente repositorio 

\verb"https://github.com/finiterank/saber_notebooks"

Se encuentra el archivo \verb"rank.2011.csv" que tiene resultados
sobre las pruebas Saber en todos los municipios de Colombia para el
año 2011. 

Para este punto tambi\'en tenga en cuenta los datos de los puntos anteriores.

\begin{itemize}
  \item[a)] (40 puntos) Tomando en cuenta los resultados de las pruebas saber
    (puntajes de matem\'aticas y lenguaje), los datos de \emph{Alta
    dependencia econ\'omica}, \emph{Inasistencia Escolar}, \emph{Sin
    Fuente Agua Mejorada}. ¿Hay alguna diferencia en la distribución
    de estas variables entre los diferentes clusters definidos de
    acuerdo a los homicidios?    
\end{itemize}

\end{enumerate}

\end{document}
