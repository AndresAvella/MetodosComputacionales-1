
%--------------------------------------------------------------------
%--------------------------------------------------------------------
% Formato para los talleres del curso de Métodos Computacionales
% Universidad de los Andes
%--------------------------------------------------------------------
%--------------------------------------------------------------------

\documentclass[11pt,letterpaper]{exam}
\usepackage[utf8]{inputenc}
\usepackage[spanish]{babel}
\usepackage{graphicx}
\usepackage{tabularx}
\usepackage[absolute]{textpos} % Para poner una imagen en posiciones arbitrarias
\usepackage{multirow}
\usepackage{float}
\usepackage{hyperref}
\decimalpoint

\begin{document}
\begin{center}
{\Large Métodos Computacionales} \\
Tarea 3 --- 2018-10\\

\end{center}


\vspace{0.3cm}

\noindent
La solución a este taller debe subirse por SICUA antes de las 5:00PM
del lunes 19 de marzo del 2018. 
Si se entrega la tarea antes del lunes 12 de marzo del 2018 a las
11:59PM los ejercicios se van a calificar con el bono indicado. 
\noindent

\vspace{0.3cm}
(10 puntos) Los archivos del c\'odigo  deben subirse en un
\'unico archivo \verb".zip" con el nombre
\verb"NombreApellido_taller3.zip", por ejemplo si su nombre es Vandana
Shiva deber\'ia subir el zip
\verb"VandanaShiva_taller3.zip" al descomprimir el zip debe crearse la
carpeta \verb"VandanaShiva_taller3" y adentro deben estar los archivos
solicitados. 
En la implementaci\'on principal de los algoritmos solicitados la
copia y reutilizaci\'on de c\'odigo de cualquier fuente de internet
(inclu\'ido el repositorio del curso) deja la nota en cero.  

\begin{questions}



  \question{{\bf Un script para hacer todo}}

(10 puntos) 
Prepare un unico script llamado \verb"NombreApellido_tarea3.sh" 
con los comandos para descargar los datos y ejecutar los scripts de python
de los siguientes ejercicios.

    \question{{\bf Airbnb Berlin}}

(30 (35) puntos) En este ejercicio usted debe analizar los datos de Airbnb
    de la ciudad de Berlin que se encuentran en
    \url{https://s3.amazonaws.com/tomslee-airbnb-data-2/berlin.zip}. 
    Para esto escriba el script de Python \verb"berlin.py" que
    {\bf para cada uno de los archivos dentro del zip:}\\
    \begin{itemize}
    \item {Lea los datos de \verb"overall\_satisfaction", \verb"acommodates",
        \verb"bedrooms", \verb"price", \verb"minstay".}
    \item{Calcule la media y desviaci\'on est\'andar de cada columna
      para renormalizar los datos para que tengan media cero y
      desviaci\'on est\'andar de uno.}
    \item{Usando su propia implementaci\'on, calcule la matriz de
      covarianza.}
    \item{Utilice matriz de covarianza para calcular la lista de las
      dos variables que est\'an m\'as correlacionadas y las dos m\'as
      anticorrelacionadas.}
    \end{itemize}
    Los resultados de las variables
    correlacionadas/anticorrelacionadas se deben escribir en el
    archivo de texto \verb"variables_berlin.txt" con cinco columnas: la
    fecha que se encuentra el nombre del archivo procesado (por
    ejemplo \verb"2015-07-04"), el nombre
    de las dos variables correlacionadas y el nombre de las dos
    variables anticorrelacionadas. 



\question {{\bf Educaci\'on, Etnicidad y Salarios}}

(30 (35) puntos) En este ejercicio usted debe analizar los datos de
educaci\'on, etnicidad y salarios que se encuentran aqu\'i:

\url{https://github.com/vincentarelbundock/Rdatasets/blob/master/csv/DAAG/cps1.csv}.

La descripci\'on de estos datos se encuentra aqui:

\url{https://github.com/vincentarelbundock/Rdatasets/blob/master/doc/DAAG/cps1.html}

    Para esto escriba el script de Python \verb"pca_salario.py" que
    incluya una funci\'on

    \verb"def predice_salario(age,educ,black,hisp,marr,nodeg)" 

    que
    prediga el salario en 1974 dadas las entradas correspondientes. 
    La funci\'on debe devolver un float con la predicci\'on para el
    salario.
    La predicci\'on se {\bf debe} calcular usando an\'alisis de componentes
    principales tal como se explica en la secci\'on 6.3.1 del libro ISL.

\question {{\bf Spline c\'ubico}}

(30 (35) puntos) El spline c\'ubico es un m\'etodo de interpolaci\'on. La idea es que dados
$N$ pares de puntos discretos ($x_0,\ldots\, x_{N-1}$) con sus valores
correspondientes ($y_0\ldots, y_{N-1}$) se puede encontrar un conjunto
de funciones que conecta esos puntos de una manera suave.
En este ejercicio debe escribir su propia implementaci\'on tomando
como condici\'on de frontera que la segunda derivada en los puntos
extremos es cero.

Para esto debe crear un archivo llamado \verb"interpolacion.py" que
contenga a la funci\'on 

\verb"def mi_spline(x_in, y_in, x_inter)" 

que
toma como entrada los arrays de numpy \verb"x_in", \verb"y_in" como los
puntos que dan la base para crear la interpolaci\'on y \verb"x_inter"
los puntos que se desean interpolar. 

Si \verb"x_inter" es un escalar
la funci\'on debe devolver un escalar con el valor de la
interpolaci\'on en ese punto.
Si \verb"x_inter" es un array de numpy la funci\'on debe devolver un
array de numpy con los valores correspondientes de la interpolaci\'on
en esos puntos.

La funci\'on \verb"mi_spline" {\bf debe} encontrar los valores buscados
a trav\'es de la resoluci\'on del sistema de ecuaciones lineales que
definen al spline c\'ubico.
Ver por ejemplo \url{https://en.wikiversity.org/wiki/Cubic_Spline_Interpolation}

\end{questions}



\end{document}
