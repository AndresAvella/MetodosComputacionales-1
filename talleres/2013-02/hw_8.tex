\documentclass{article}
\title{Taller \#8. F\'isica Computacional / FISI 2025 \\Semestre
  2013-II. \\ Profesor: Jaime E. Forero Romero} 
\date{Noviembre 7, 2013}
\begin{document}
\maketitle

{\bf Esta tarea debe resolverse por parejas (i.e. grupos de 2
  personas) y la respuesta debe estar en un repositorio de github con
  un commit final antes del medio d\'ia del lunes 18 de noviembre del
  2013.} 

El objetivo es usar MCMC para encontrar los par\'ametros de un modelo
f\'isico que describe la evoluci\'on temporal de un sustrato
fosforilado. Esto es parte de la tesis de pregrado {\bf Modulation of
  the thermodynamic behavior in the membrane of Bacillus subtilis by
  the activity of the enzyme Desaturase $\Delta$-5 des} de Juanita
Lara. 

Una parte clave en este trabajo requiere seguir la evoluci\'on
temporal de un sustrato $P$. Esta evoluci\'on est\'a descrita por la
siguiente ecuaci\'on diferencial:

\begin{equation}
\frac{dP}{dt} = \frac{\alpha(S_0 -P)}{K_{m1}+S_0-P} - \frac{\beta
  P}{K_{m2}+P}, 
\end{equation}

donde $\alpha$, $\beta$, $K_{m1}$, $K_{m2}$ y $S_0$ son constantes. Una
descripci\'on detallada de los fundamentos detr\'as de esta ecuaci\'on
puede encontrarse en el repositorio en
\verb"homework/draft_juanita.pdf". 

Juanita midi\'o en el laboratorio la evoluci\'on de $P$ en funci\'on
del tiempo. Estos datos se encuentran en el repositorio del curso en
un archivo llamado \verb"homework/dimer_observations.dat".

El objetivo es encontrar el valor m\'as probable de las constantes
$\alpha$, $\beta$, $K_{m1}$, $K_{m2}$ y $S_0$. 

\begin{enumerate}
\item Escriba un programa en python que realice una exploraci\'on MCMC
  del espacio de par\'ametros relevante para el problema.

\item Prepare gr\'aficas de contorno de la distribuci\'on de $chi^2$
  en planos bidimensionales que sean las combinaciones de los
  par\'ametros. Ejemplo: contornos de la distribuci\'on de $\chi^2$ en el
  plano $\alpha$-$\beta$.

\item Prepare gr\'aficas que comparen los datos observacionales con
  los resultados de la ecuaci\'on diferencial evaluada con los mejores
  par\'ametros encontrados por el c\'odigo MCMC.

\item Eval\'ue el intervalo de incertidumbre para las constantes del
  modelo. 



\end{enumerate}
El programa debe poder ejecutarse y preparar las gr\'aficas con
  un archivo Makefile.

\end{document}
