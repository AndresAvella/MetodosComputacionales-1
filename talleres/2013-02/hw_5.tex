\documentclass{article}
\title{Taller \#5. F\'isica Computacional / FISI 2025 \\Semestre
  2013-II. \\ Profesor: Jaime E. Forero Romero} 
\date{Octubre 4, 2012}
\begin{document}
\maketitle

{\bf Esta tarea debe resolverse por parejas (i.e. grupos de 2
  personas) y debe estar en un repositorio de la cuenta de github de
  uno de los miembros de cada equipo con un commit final hecho antes del
  medio d\'ia del jueves 17 de Octubre del 2013. El esquelto del
  c\'odigo debe estar en el repositorio antes del medio d\'ia del
  jueves 10 Octubre  del 2013.}   

El objetivo de la tarea es escribir un c\'odigo en Python (o en un
notebook the IPython) y escribir un peque\~no informe sobre los
siguientes puntos.

\begin{enumerate}

\item
Dentro del repositorio de datos del curso en
\verb" homework/hw_4/sampled+ma0844az_1-1+_data.txt"  se encuentran
los datos de encefalogramas que ya se utilzaron en la clase anterior.

\begin{enumerate}

\item 
El c\'odigo debe procesar los datos de cada encefalograma  para obtener una
transformada de Fourier a partir de la definci\'on:
\begin{equation}
\hat{x}_{k} = \sum_{n=0}^{N-1}x_{n}\exp(-2\pi i k n/N), 
\end{equation}
donde $x_{n}$ es la serie de $N$ puntos en funci\'on del tiempo y los
valores de $k$ corresponden a diferentes
frecuencias. Espec\'ificamente el programa debe calcular las
frecuencias y los n\'umeros complejos correspondientes a cada
$k=0,1,\dots,N-1$. 

{\bf Importante}. No se trata de programar la suma sino de usar las
rutinas FFT para hacer el c\'alculo.


El objetivo principal es calcular el espectro de potencias, $P$, de
cada se\~nal. Es decir, calcula la norma al cuadrado de cada
$\hat{x}_{k}$ en funci\'on de la frecuencia y prepara una gr\'afica de
$P$  como funci\'on de la frecuenca $f$ para todas las se\~nales. El
espacio entre cada medici\'on del encefalograma es de 1 minuto. ¿Hay
alguna diferencia entre el espectro de potencias de cada una de las
se\~nales?  



\item 
Conserve ahora los 10 valores de $\hat{x}_{k}$ que tengan el valor
m\'as grande de us norma al cuadrado y haga cero el resto. El objetivo
es reconstruir la se\~nal original a partir de estas 10
componentes. C\'omo se comparan (por medio de una estad\'istica
$\chi^2$) la se\~nal reconstruida a partir de la se\~nal original?
\end{enumerate}


\item
Ahora tienen los datos del n\'umero de manchas solares en funci\'on
del tiempo en \verb"homework/hw_5/sparse_sample_monthrg.dat". La
primera columna corresponde al a\~no, la segunda al mes, la tercera al
n\'umero de d\'ias de datos tomados y la cuarta al promedio de
manchas. La particularidad de estos datos es que no est\'an espaciados
homogeneamente en el tiempo.
\begin{enumerate}

\item
El primero paso es usar tres m\'etodos de interpolaci\'on (constante, lineal y
c\'ubica) para conseguir un conjunto de datos equiespaciado en el
tiempo.

\item

El segundo paso es calcular y graficar los espectros de potencias para
los datos con los tres tipos de interpolaciones. Son similares o
diferentes los espectros de potencias? comente sobre las razones para
las posibles diferencias.


\item
Haga cero todos los valores de $\hat{x}_{k}$ para $k$ menores a lo que
corresponder\'ia un per\'iodo de $20$ a\~nos y $k$ mayores a lo que
corresponder\'ia a un per\'ioso de $2$ a\~nos. Con estos datos haga la
transormada de Fourier inversa para obtener una nueva serie de puntos
$x^{\prime}_{n}$. El objetivo es preparar una gr\'afica con los nuevos
puntos y los datos originales, todo en una misma escala de n\'umero
de manchas solares como funci\'on del tiempo en unidades de a\~nos.
Los puntos $x^{\prime}_{n}$ deben tener entonces la normalizaci\'on
adecuada. 

\item 
Finalmente, el programa debe hacer una estimaci\'on del per\'iodo del
ciclo solar a partir de los datos anteriores.


\end{enumerate}
\end{enumerate}


\end{document}
