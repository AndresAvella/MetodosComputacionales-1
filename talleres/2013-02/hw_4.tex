\documentclass{article}
\title{Taller \#4. F\'isica Computacional / FISI 2025 \\Semestre 2013-II. \\ Profesor: Jaime E. Forero Romero}
\date{Septiembre 10 2013}
\begin{document}
\maketitle

{\bf Los dos programas con el c\'odigo de esta tarea deben ser subidos a la p\'agina de sicuaplus del curso como un \'unico archivo tar antes de las 1PM del Jueves 19 de Septiembre.}

\begin{enumerate}


\item (50 puntos) Escriba un programa en C que haga un fit de una
  funci\'on polinomial $f(x)=\sum_{n=0}a_nx^n$ a una serie de puntos
  $x_i, f_i$. El programa debe usar el formalismo matricial visto en
  clase. 
El programa debe compilarse con las librerias \verb"-lm -lgsl -lgslcblas".

El nombre del c\'odigo fuente es \verb"NombreApellido_polinomio.c" y el programa debe poder ejecutarse de la siguiente manera:

\begin{verbatim}
./NombreApellido_polinomio.x [input_file] [degree]
\end{verbatim}

Donde el archivo \verb"input_file" contiene dos columnas y un numero
indeterminado de filas. Cada fila corresponde a un par $x_i, f_i$,
donde $1\leq i\leq N$, siendo $N$ el n\'umero total de puntos
observados. \verb"degree" correspondo al grado del polinomio que se
debe ajustar. Este debe ser un entero $>0$. 
El output producido deben ser tantas l\'ineas como el grado del polinomio. Cada una de las l\'ineas corresponde al valor del coeficiente $a_n$. La \'ultima l\'inea corresponde al $\chi^2$ reducido del ajuste, definido como:

\begin{equation}
\chi^2 = \frac{1}{N}\sum_{i=0}^{N}(f_i - f(x_i))^2.
\end{equation}

De esta manera, un ejemplo de salida del programa es:

\begin{verbatim}
a_0 3.2
a_1 1.4
a_2 -0.4
a_3 0.3
chi2 12.0
\end{verbatim}

\item (50 puntos) Escriba un programa en C que haga un an\'alisis de
  componentes principales del archivo prueba que se encuentra en el
  repositorio \verb"ComputationalPhysicsUniandesData/" bajo el
  directorio \verb"homework/hw_4/" con el nombre
  \verb"sampled+ma0844az_1-1+_data.txt".  

El programa debe compilarse con las librerias \verb"-lm -lgsl -lgslcblas".

El nombre del c\'odigo fuente es \verb"NombreApellido_pca.c" y el
programa debe poder ejecutarse de la siguiente manera: 

\begin{verbatim}
./NombreApellido_pca.x [filename]
\end{verbatim}

El archivo de entrada \verb"filename" corresponde a series de tiempo
de un encefalograma tomado sobre un paciente real. Cada columna
representa una se\~nal de un electrodo diferente, mientras que cada
fila corresponde a un instante de tiempo. El n\'umero de columnas
siempre sera el mismo que en el archivo de prueba   \verb"sampled+ma0844az_1-1+_data.txt".  

El programa debe producir dos archivos de texto. El primero de nombre
\verb"NombreApellido_eigenvalues.dat"donde todos los autovalores
est\'an escritos. El segundo \verb"NombreApellido_eigenvectors.dat"
donde est\'an escritos los 10 primeros autovectores (cada vector en una
columna). El tercero de nombre
\verb"NombreApellido_pca_chisquare.dat" donde se escribe el $\chi^2$
entre cada se\~nal y la reconstrucci\'on correspondiente usando los 10
primeros autovectores.


\end{enumerate}

Los valores de cada punto se reparten de la manera siguiente: $10$
puntos por tener c\'odigo que compila. El resto de puntos se asignan
por tener un programa que funciona correctamente con par\'ametros de
entrada decididos por el profesor. Si el tiempo de ejecuci\'on de un
programa es mayor a 12 horas, se califica con cero puntos el ejercicio
correspondiente. 

\end{document}
