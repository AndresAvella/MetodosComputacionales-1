\documentclass{article}
\title{Taller \#6. F\'isica Computacional / FISI 2025 \\Semestre
  2013-II. \\ Profesor: Jaime E. Forero Romero} 
\date{Octubre 17, 2013}
\begin{document}
\maketitle

{\bf Esta tarea debe resolverse por parejas (i.e. grupos de 2
  personas) y debe estar en un repositorio de la cuenta de github de
  uno de los miembros de cada equipo con un commit final hecho antes del
  medio d\'ia del jueves 7 de Noviembre del 2013. El esqueleto del
  c\'odigo debe estar en el repositorio antes del medio d\'ia del
  martes 29 de Octubre del 2013.}   


El objetivo de este taller es escribir un integrador Runge-Kutta de
cuarto orden para estudiar la interacci\'on de dos galaxias de
disco. 

Vamos a seguir el ejemplo  del paper cl\'asico de Toomre \& Toomre
\textit{Galactic Bridges and Tails} en el Astrophysical Journal, Vol. 178,
pp. 623-666 (1972), el cual se encuentra repositorio como
\verb"homework/TT.pdf" 

Una galaxia ser\'a descrita como una masa central con 120
cuerpos en \'orbitas circulares que la rodean. Una imagen que puede
servir es la del sistema solar, pero donde solamente existen 5 orbitas
circulares posibles y en cada \'orbita se encuentran 12, 18, 24, 30 y 36
cuerpos, tal como se muestra en la Figura 1 del paper, en el panel
marcado con "$-1$". 

El movimiento {\bf de cada una} de las part\'iculas que rodean a la masa
central est\'a determinado por la siguiente ecuaci\'on diferencial
ordinaria vectorial de segundo orden: 
 
\begin{equation}
\frac{d^2\vec{r}}{dt^2} = -\frac{GM}{r^2} \hat{r}
\end{equation}

donde $\vec{r}$  es un vector que va de la masa central a la
part\'icula, $\hat{r}$ es el vector unitario correspondiente,
$r=|\vec{r}|$, $M$ es la masa del cuerpo central y $G$ es la constante
de gravitaci\'on.  Una vez se conocen las posiciones y velocidades
iniciales de cada una de las masas $\vec{r}_0$, $\vec{v}_0$ es posible
conocer la posici\'on y la velocidad en momentos siguientes. Noten que
estamos usando una  aproximaci\'on donde la fuerza que siente cada
part\'icula solamente se debe solamente a la masa central. Esto es
equivalente a decir a  que la masa de todas las part\'iculas es
despreciable con respecto a la masa central. As\'i mismo la masa central no siente ninguna  fuerza apreciable por parte de las part\'iculas.

\begin{enumerate}
\item El primer punto de la tarea consiste escribir un programa que genere
  las condiciones  iniciales (posiciones y velocidades) para que las
  100 part\'iculas de una galaxia de disco orbiten de manera estable
  en c\'irculos alrededor de la masa  central. La masa del cuerpo
  central debe ser de $10^{12}$ masas solares, el radio externo de
  $50$ kiloparsecs y cada una de las orbitas circulares debe estar
  equiespaciada a $10$ kiloparsecs (1 parsec son
  $3.0\times10^{16}$metros).  El centro de la galaxia debe ubicarse en
  el punto $x_{0}\hat{i} + y_{0}\hat{j}$ kpc y la velocidad del centro de
  masa es $v_x\hat{i}+v_y\hat{j}$ km/s y donde $x_{0}$, $y_{0}$,
  $v_{x}$ y $v_{y}$ son
  n\'umeros arbitrarios que se deben como argumento del c\'odigo al
  momento de ser ejecutado. Estas condiciones iniciales (posiciones y
  velocidades) deben ser
  escritas en un archivo de texto al momento de ejecutar el
  c\'odigo. En este archivo la primera columna ser\'a un entero que
  llamaremos ID y corresponde a un n\'umero entero de cada
  part\'icula. El ID negativo si se trata de la part\'icula
  central, y cero o positivo si se trata de las part\'iculas en las \'orbitas
  circulares. Pista: ¿Cu\'al es la aceleraci\'on centr\'ipeta de una
  part\'icula en una  \'orbita de radio $r$? 
\item 
  Escriba el c\'odigo que evolucione la posici\'on y la velocidad de
  cada una de las part\'iculas durante 5 mil millones de
  a\~nos.  Este c\'odigo debe leer las condiciones iniciales generadas
  por el c\'odigo anterior. Se debe utilizar un m\'etodo de
  Runge-Kutta de cuarto orden para integrar la ecuaci\'on (1) para
  cada part\'icula del disco.  El c\'odigo debe generar cinco archivos
  de texto con el entero de identidad, posiciones, velocidades de las part\'iculas en 5  momento diferentes equiespaciados en los 5 mil millones de a\~nos de
  evoluci\'on del sistema. El c\'odigo debe tomar como argumento el
  nombre del archivo con las condiciones iniciales. 
\item 
  Prepare un programa de python que toma como entrada un n\'umero arbitario de 
  archivos generados por el c\'odigo anterior y prepara gr\'aficas que
  muestran las posiciones.  Utilice estre programa para mostrar que
  las part\'iculas que representan el disco de la galaxia siguen en su
  \'orbita circular despu\'es de 5 mil millones de a\~nos. 
\item 
  Este punto incluye una segunda galaxia id\'entica a la
  primera. Si se considera que la galaxia anterior tiene un centro de
  masa en la posici\'on $(0\hat{i}+0\hat{j})$ kpc y una velocidad de centro de
  masa nula, vamos a considerar ahora que la masa central de la
  segunda galaxia tiene una posicion $(150\hat{i}+200\hat{j})$ kpc y
  el centro de masa  tiene una velocidad inicial $-100\hat{j}$ km/s. La
  direcci\'on $\hat{z}$ es perpendicular al plano del disco. 
  Las condiciones iniciales de cada una de las galaxias son generadas
  con el c\'odigo del primer punto. Concatenando los dos archivos
  (utilizando el comando \verb"cat") debe generar un archivo \'unico
  de condiciones iniciales.
  
  Ahora
  evolucione estas nuevas condiciones iniciales por 5 mil millones de
  a\~nos para ver la interacci\'on de las dos galaxias.  Prepare
  gr\'aficas de la posicion de las part\'iculas de las dos galaxies en
  5 momentos diferentes equiespaciados en los  5mil millones de a\~nos
  de evoluci\'on del sistema. 
  Note que en esta configuraci\'on las masas centrales sienten su
  influencia mutua y su \'orbita tambi\'en debe ser calculada.

\item
  Ahora el programa debe poder calcular la colision de las galaxias
  cuando se generan las condiciones iniciales para dos galaxias con
  posiciones y velocidades del centro de masa arbitrarios generados
  con el programa de condiciones iniciales.

\end{enumerate}

El repositorio debe tener al menos tres c\'odigos con los siguientes nombres:

\begin{enumerate}
\item \verb"IC.c": c\'odigo que genera las condiciones iniciales de
  una sola galaxia con las posiciones y velocidades del centro de masa
  dadas como par\'ametros de entrada al momento de ejecutar el programa.
\item \verb"evolve.c": c\'odigo que evoluciona un archivo con 
  condiciones iniciales dadas al momento de ejecutar el programa.
\item \verb"plots.py": c\'odigo que prepara gr\'aficas de
  posiciones, tomando como entrada un n\'umero
  arbitrario de archivos de ID, posiciones y velocidades.
\end{enumerate}


En la calificaci\'on se dar\'a un 20\% a cada uno de los puntos del 1
al 5. Solamente se recibir\'an tareas que est\'en en un repositorio de
github. Las gr\'aficas deben preparse con programas en python. La
respuesta a cada uno de los puntos debe ser redactada en un informe
que debe estar dentro del repositorio. Todos los pasos deben poder
ejecutarse con un Makefile.



Divi\'ertanse: \verb"http://viz.adrian.pw/galaxy/"

\end{document}
