\documentclass{article}
\title{Taller \#3. F\'isica Computacional / FISI 2025 \\Semestre 2013-II. \\ Profesor: Jaime E. Forero Romero}
\date{Agosto 20 2013}
\begin{document}
\maketitle

{\bf Los cuatro programas con el c\'odigo de esta tarea deben ser subidos a la p\'agina de sicuaplus del curso como un \'unico archivo tar antes de las 1PM del Jueves 29 de Agosto.}

\begin{enumerate}


\item (20 puntos) Escriba un programa que cuente el numero de vocales (a,e,i,o,u) en un archivo UTF-8 arbitrario y que imprima en pantalla una l\'inea indicando la vocal y el n\'umero de veces que fu\'e encontrada. Se deben tomar en cuenta las mayusculas y los casos con tilde (solamente en espa\~nol).

El nombre del archivo del c\'odigo debe ser \verb"NombreApellido_vocales.py" y el programa debe poder ejecutarse de la siguiente manera:

\begin{verbatim}
python NombreApellido_vocales.py [input_file]
\end{verbatim}

El output producido deben ser cinco l\'ineas con las cuentas de cada vocal. Ejemplo:

\begin{verbatim}
a 34095
e 1233
i 12399
o 9455
u 1484
\end{verbatim}

\item (20 puntos) Escriba un programa que diga si un n\'umero entero menor o igual a $10^6$ puede descomponerse como multiplicaci\'on de dos factores primos. Si es posible, que imprima los dos factores en pantalla. Si no es posible, o el valor de entrada es incorrecto (es un n\'umero negativo o major que $10^6$), el programa debe imprimir un mensaje indicando por qu\'e la separaci\'on en dos factores primos no se puede hacer.


El nombre del archivo del c\'odigo debe ser \verb"NombreApellido_primos.py" y el programa debe poder ejecutarse de la siguiente manera:

\begin{verbatim}
./python NombreApellido_primos.py [numero_para_descomponer]
\end{verbatim}

Si la descomposici\'on entre n\'umeros primos es posible. El output debe ser de una l\'inea con los dos n\'umeros primos. Ejemplo,

\begin{verbatim}
509 997
\end{verbatim}

\item (30 puntos) Escriba un programa que diga si una serie de frases escritas dentro de un archivo de texto son pal\'indromos.

El nombre del archivo del c\'odigo debe ser \verb"NombreApellido_pali.py" y el programa debe poder ejecutarse de la siguiente manera:

\begin{verbatim}
./python NombreApellido_pali.py [inputfile]
\end{verbatim}

Si dentro del archivo se encontraran las frases:
\begin{verbatim}
Amar de traer cama: ama crearte drama.
Avid as a fool; aloof as a diva.
Dabale arroz la zorras al abad.
Desearte - reconocere -– trae sed.
\end{verbatim}

La salida debe ser
\begin{verbatim}
True
True
False
True
\end{verbatim}

\item (30 puntos) Escriba un programa que liste los archivos en el directorio donde se encuentra y cuenta el n\'umero de lineas que contiene cada archivo.

El nombre del archivo del c\'odigo debe ser \verb"NombreApellido_wc.py" y el programa debe poder ejecutarse de la siguiente manera:

\begin{verbatim}
./python NombreApellido_wc.py
\end{verbatim}

La salida debe ser del siguiente estilo
\begin{verbatim}
NombreApellido_wc.py 40
Metamorfosis.txt 503450
compras.txt 8
\end{verbatim}


\end{enumerate}

Los valores de cada punto se reparten de la manera siguiente: $10$ puntos por tener c\'odigo que se ejecuta sin errores. El resto de puntos se asignan por tener un programa que funciona correctamente con par\'ametros de entrada decididos por el profesor.

\end{document}
