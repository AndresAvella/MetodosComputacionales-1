\documentclass{article}
\title{Taller \#4. F\'isica Computacional / FISI 2025 \\Semestre
  2013-I. \\ Profesor: Jaime E. Forero Romero} 
\date{Marzo 14 2013}
\begin{document}
\maketitle

{\bf Esta tarea debe resolverse por parejas (i.e. grupos de 2
  personas) y debe estar en un repositorio de la cuenta de github de
  uno de los miembros de cada equipo con un commit final hecho antes del
  medio d\'ia del jueves 21 de Marzo del 2013}  

Desde hace cerca de 400 a\~nos existen observaciones sistem\'aticas del
n\'umero de manchas solares. En la \'epoca moderna se ha confirmado
que la cantidad de estas manchas est\'a correlacionada con un aumento
de actividad en el Sol en t\'erminos de ejecciones de masa y
part\'iculas cargadas que pueden ser peligrosas para la vida humana en
la Tierra.

Dentro del repositorio del curso en \verb"homework/hw5_data"
encontrar\'an el archivo \verb"monthrg.dat" que incluye datos
mensuales de n\'umeros de manchas solares observadas desde 1610. El
formato del archivo est\'a descrito dentro del \verb"README" en el
mismo directorio.

El objetivo de la tarea es escribir un c\'odigo en Python (o en un
notebook the IPython) que haga las siguientes tareas:
\begin{enumerate}

\item
Prepare una gr\'afica de n\'umero de manchas solares en funci\'on del
tiempo. Donde el eje $x$ tiene unidades de a\~nos.

\item
Procesa los datos anteriores (n\'umero de manchas en funci\'on del
tiempo) para obtener una transformada de Fourier a partir de la
definci\'on:
\begin{equation}
\hat{x}_{k} = \sum_{n=0}^{N-1}x_{n}\exp(-2\pi i k n/N), 
\end{equation}
donde $x_{n}$ es la serie de $N$ puntos en funci\'on del tiempo y los
valores de $k$ corresponden a diferentes
frecuencias. Espec\'ificamente el programa debe calcular las
frecuencias y los n\'umeros complejos correspondientes a cada
$k=0,1,\dots,N-1$. 

{\bf Importante}. No se trata de programar la suma sino de usar las
rutinas FFT para hacer el c\'alculo.

\item

Calcula el espectro de potencias, $P$. Es decir, calcula la norma al
cuadrado de cada $\hat{x}_{k}$ en funci\'on de la frecuencia y prepara
una gr\'afica de $P$  como funci\'on de la frecuenca $f$. 

\item 
Prepara una gr\'afica del espectro de potencias esta vez como
funci\'on del periodo $T=1/f$  para per\'iodos entre 1 y 20
a\~nos. Alcanza a ver que algun valor del espectro de potencias domina
para un per\'iodo determinado? 

\item 
Hace cero todos los valores de $\hat{x}_{k}$ para $k$ mayores a lo que
corresponder\'ia un per\'iodo de $20$ a\~nos y hace la transormada de
Fourier inversa para obtener una nueva serie de puntos
$x^{\prime}_{n}$. El objetivo es preparar una gr\'afica con los nuevos
puntos y los datos originales, todo en una misma escala de n\'umero
de manchas solares como funci\'on del tiempo en unidades de a\~nos.
Los puntos $x^{\prime}_{n}$ deben tener entonces la normalizaci\'on
adecuada. 

\item 
Finalmente, el programa debe hacer una predicci\'on de cu\'ando se
dar\'a el siguiente m\'aximo solar a partir de los resultados de los
puntos anteriores.

\item
Enviar un email al monitor del curso Daniel Felipe Duarte {\tt
  df.duarte578} en {\tt uniandes.edu.co} con el subject
\verb"RESPUESTA TALLER 5 FISICA COMPUTACIONAL". En el cuerpo del texto
debe ir la direcci\'on del repositorio donde est\'a la tarea. 


\end{enumerate}

En la calificaci\'on se dar\'a un 20\% a cada uno de los puntos del
2 al 6.

\end{document}
