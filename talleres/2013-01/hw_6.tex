\documentclass{article}
\title{Taller \#6. F\'isica Computacional / FISI 2025 \\Semestre
  2013-I. \\ Profesor: Jaime E. Forero Romero} 
\date{Abril 9 2013}
\begin{document}
\maketitle

{\bf Esta tarea debe resolverse por parejas (i.e. grupos de 2
  personas) y debe estar en un repositorio de la cuenta de github de
  uno de los miembros de cada equipo con un commit final hecho antes del
  medio d\'ia del martes 16 de Abril del 2013}  

Se les presenta el siguiente sistema de ODEs acopladas:

\begin{equation}
\frac{dx}{dt} = \sigma(y-x)
\end{equation}

\begin{equation}
\frac{dy}{dt} = x(\rho -z) - y
\end{equation}

\begin{equation}
\frac{dz}{dt} = xy -\beta z
\end{equation}

donde $\sigma$, $\rho$ y $\beta$ son constantes.

\begin{enumerate}
\item 
Escriba un programa en $C$ que resuelva estas ecuaciones diferenciales para $\sigma=10$, $\beta=8.0/3.0$ y $\rho=28.0$ mediante un m\'etodo Runge-Kutta de cuarto orden para condiciones iniciales arbitrarias $x_0,y_0,z_0$ en intervalo de tiempo $0.0<t<3.0$. El programa debe estar escrito en al menos dos archivos de c\'odigo fuente separados. Uno debe contener el \verb"main" y los otros archivos las rutinas que considere necesarias para ejecutar el programa. 

\item 
Ejecute el programa para $10$ condiciones iniciales diferentes donde cada  $x_0,y_0,z_0$ es un n\'umero aleatorio entre $-10$ y $10$. Presente los resultados para esas $10$ condiciones iniciales en tres gr\'aficas bidimensionales que presenten las orbitas en los planos $x-y$, $x-z$ y $y-z$.

\item 
Haga un archivo \verb"Makefile" que: compile el c\'odigo, lo ejecute y produzca las gr\'aficas cuando se ejecute el comando \verb"make" en el directorio que contiene el c\'odigo fuente.

\end{enumerate}

En la calificaci\'on se dar\'a un 33\% a cada uno de los puntos del 1 al 3. Solamente se recibir\'an tareas que est\'en en un repositorio de github.


Enviar un email al monitor del curso Daniel Felipe Duarte {\tt
  df.duarte578} en {\tt uniandes.edu.co} con el subject
\verb"RESPUESTA TALLER 6 FISICA COMPUTACIONAL". En el cuerpo del texto
debe ir la direcci\'on del repositorio donde est\'a la tarea. 



\end{document}
