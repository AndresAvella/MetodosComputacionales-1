\documentclass{article}
\usepackage{hyperref}
\textheight=25.5cm
\textwidth=16.0cm
\oddsidemargin=-0.5cm
\topmargin=-3.0cm
\usepackage[pdftex]{graphicx}
\usepackage[utf8]{inputenc}
\usepackage[spanish]{babel}
 \usepackage{amsmath}
\title{Taller \#8 de M\'etodos Computacionales\\ FISI 2028, Semestre 2014 - 20}
\author{Profesor: Jaime Forero}
\date{Martes 11 de Noviembre del 2014}
\begin{document}
\maketitle
\thispagestyle{empty}


{\bf Importante}
\begin{itemize}

\item Todo el c\'odigo fuente y los datos se deben encontrar en un
  repositorio en github con un commit final hecho antes del medio
  d\'ia lunes 24 de noviembre. El nombre del repositorio debe ser
  \verb"Apellidos1_Apellidos2_hw8", por ejemplo si trabajo con
  Nicol\'as deber\'iamos crear un repositorio llamado
  \verb"Forero_Garavito_hw8".  

\item 
  La nota m\'axima de este taller es de 100 puntos. Se otorgan 1/3
  de los puntos si el c\'odigo fuente es razonable, 1/3 si se puede
  compilar/ejecutar y 1/3 si da los resultados correctos.  

\item
  El primer problema es obligatorio para todos los grupos. Para
  completar 100 puntos se debe elegir uno de los otros dos problemas. 

\item
  Las respuestas a todos los puntos se deben poder ver y ejecutar
  desde un notebook de  Ipython.  
\end{itemize}


\begin{enumerate}

\item {\bf El perfil radial de las ves\'iculas} (50 puntos)

Vamos a volver a trabajar con las im\'agenes de ves\'iculas en formato
TIFF que se encuentran en
\url{https://github.com/forero/ComputationalMethodsData/tree/master/homework/hw_8/}.

El objetivo es encontrar el centro y los puntos que describen el
contorno de las ves\'iculas. Para esto vamos a tomar que desde el
centro de la ves\'icula el contorno cumple la siguiente ecuaci\'on en
coordenadas polares:

\begin{equation}
r(\theta) = r_{0} + r_{1}\sin(m\theta + \phi), 
\label{eq:1}
\end{equation}
donde $r_0$, $r_1$, $\phi$ y $m$ son par\'ametros libres, con $m>0$ entero.

Deben escribir un programa que a trav\'es de MCMC encuentre las
distribuciones de probabilidad de la posici\'on del centro del
c\'irculo y de los par\'ametros $r_0$, $r_1$, $\phi$ y $m$. 
Para esto la idea es que en cada paso se genere una imagen para ser
comparada con la imagen original y poder calcular el $\chi^2$. 


\item {\bf El hessiano de la intensidad de las ves\'iculas} (50
  puntos)

Volviendo a trabajar con los mismos datos del punto anterior, el
objetivo es el mismo pero esta vez con un m\'etodo diferente.

Calculen el Hessiano de la intensidad de la imagen en cada pixel y
encuentren los autovalores y autovectores correspondientes. ¿Qué
propiedades tienen los autovalores y autovectores que marcan el borde
de la ves\'icula?

Escriba un programa que encuentre los pixeles que marcan el borde de
la ves\'icula y haga un fit de estos puntos con la funci\'on de la
Ecuaci\'on (\ref{eq:1}).

\item {\bf Tensor de deformaci\'on despu\'es de un terremoto} (50 puntos)

En el mismo repositorio de los datos anteriores se encuentran los
datos de desplazamiento de 32 estaciones despu\'es del terremoto de
Antofagasta en 1995.

El objetivo es estimar la parte simétrica del tensor de deformaci\'on
en la posición de cada una de las estaciones y graficar la magnitud
del autovalor principal y la dirección del autovector principal
correspondiente. ¿Cómo se pueden interpretar estos resultados? 

\end{enumerate}


\end{document}
