\documentclass{article}
\usepackage{hyperref}
\textheight=25.5cm
\textwidth=16.0cm
\oddsidemargin=-0.5cm
\topmargin=-3.0cm
\usepackage[pdftex]{graphicx}
\usepackage[utf8]{inputenc}
\usepackage[spanish]{babel}
\title{Taller \#3 de M\'etodos Computacionales\\ FISI 2028, Semestre 2014 - 20}
\author{Profesor: Jaime Forero}
\date{Viernes 29 de Agosto, 2014}
\begin{document}
\maketitle
\thispagestyle{empty}


{\bf Importante}
\begin{itemize}

\item Los dos archivos de c\'odigo fuente de soluci\'on de esta tarea
  deben subirse a trav\'es de sicuaplus antes de las medio d\'ia del
  viernes 12 de Septiembre como un \'unico archivo zip con el nombre
  \verb"NombreApellidos_hw3.zip", por ejemplo yo deber\'ia subir un
  archivo llamado \verb"JaimeForero_hw3.zip" 

\item 
La nota m\'axima de este taller es de 100 puntos. Se otorgan 1/3
  de los puntos si el c\'odigo fuente es razonable, 1/3 si se puede
  compilar/ejecutar y 1/3 si da los resultados correctos.  
\item
Si se entrega la tarea antes del medio d\'ia del 5 de septiembre los
puntos se calificar\'an sobre 55, as\'i que la nota m\'axima puede ser
de 110. 
\end{itemize}

\begin{enumerate}


\item {\bf Puntos de Lagrange}
(50 puntos)

Escriba un programa llamado \verb"lagrange.py" en Python que encuentre
los puntos de
Lagrange\footnote{\url{http://es.wikipedia.org/wiki/Puntos_de_Lagrange}}
del sistema Tierra-Sol. Los puntos se deben
  encontrar num\'ericamente, no usando soluciones anal\'iticas. 
  
  Los resultados se deben guardar en un archivo de nombre
  \verb"puntos_lagrange.dat"  Los puntos deben estar escritos en dos
  columnas correspondientes a las posiciones $x$ y $y$, donde se toma
  al Sol como ubicado en el punto $(0,0)$ y a la Tierra en $(1,0)$. 


\item {\bf De nuevo, marcha aleatoria}
(50 puntos) 

Escriba un programa en Python que resuelva el
  problema 3 del Taller 2. El programa se debe llamar
  \verb"marcha_3D.py". El reto es lograr reescribir el c\'odigo de tal
  manera que no se demore horas en correr. 


\end{enumerate}

\end{document}
