\documentclass{article}
\usepackage{hyperref}
\textheight=25.5cm
\textwidth=16.0cm
\oddsidemargin=-0.5cm
\topmargin=-3.0cm
\usepackage[pdftex]{graphicx}
\usepackage[utf8]{inputenc}
\usepackage[spanish]{babel}
\title{Taller \#6 de M\'etodos Computacionales\\ FISI 2028, Semestre 2014 - 20}
\author{Profesor: Jaime Forero}
\date{Martes 21 de Octubre, 2014}
\begin{document}
\maketitle
\thispagestyle{empty}


{\bf Importante}
\begin{itemize}

\item Todo el c\'odigo fuente y los datos se debe encontrar en un
  repositorio en github con un commit final hecho antes del medio
  d\'ia del viernes 31 de Octubre. El nombre del repositorio debe ser
  \verb"Apellidos1_Apellidos2_hw6", por ejemplo si trabajo con
  Nicol\'as deber\'iamos crear un repositorio llamado
  \verb"Forero_Garavito_hw6".  

\item 
  La nota m\'axima de este taller es de 100 puntos. Se otorgan 1/3
  de los puntos si el c\'odigo fuente es razonable, 1/3 si se puede
  compilar/ejecutar y 1/3 si da los resultados correctos.  

\item
  El miercoles 29 de octubre tendremos una sesi\'on para presentar
  avances de diferentes grupos en cada una de las preguntas del
  taller. Hay un bono de 15 puntos para cada grupo que presente un
  avace significativo.  


\end{itemize}


\begin{enumerate}


\item {\bf Din\'amica de cazadores y de presas} (30 puntos)


Tenemos dos especies, una caza a la otra. El n\'umero de presas es $x$
y el numero de cazadores es $y$. Su evoluci\'on temporal est\'a
descrita por las siguientes dos ecuaciones diferenciales acopladas

\begin{equation}
\frac{dx (t)}{dt} = Ax(t) - Bx(t)y(t),
\end{equation}

\begin{equation}
\frac{dy (t)}{dt} = -Cy(t) + Dx(t)y(t).
\end{equation}

\begin{itemize}
\item Encuentre los valores de $x(0)$ y $y(0)$ para que exista equilibrio
($\dot{x}=\dot{y}=0$) asumiendo que $A=20$, $B=1$, $C=30$ y $D=1$.

\item (10 puntos) Escriba un c\'odigo en C que solucione este sistema acoplado de
  ecuaciones para $0<t<1$ y escriba los resultados ($t,x,y$) en un
  archivo llamado   \verb"poblaciones_x0_y0.dat" donde \verb"x0",
  \verb"y0" son los   valores iniciales utilizados para la
  integraci\'on. El c\'odigo de   debe poder ejecutar como 
\begin{verbatim}
./runge_kutta.x x0 y0
\end{verbatim}

\item Corra varias veces el c\'odigo variando las
  condiciones iniciales de la siguiente manera: $y(0)$ se mantiene
  fijo en el valor encontrado en el punto anterior mientras que $x(0)$
  disminuye en una unidad con respecto al valor encontrado en el punto
  anterior hasta llegar a 1. Explore las diferentes soluciones.

\item (10 puntos)  Escriba un c\'odigo en Python que se pueda
  ejecutar de la siguiente manera
  \verb"plot_poblaciones.py poblaciones_x0_y0.dat" y cree un pdf
  \verb"poblaciones_x0_y0.pdf" con una gr\'afica en el plano $x-y$ de
  los puntos del archivo de datos 
  correspondiente. 

\item (10 puntos) Escriba un \verb"makefile" que enlace correctamente
  todos los pasos anteriores y cree todas las gr\'aficas para las
  diferentes condiciones iniciales. 
\end{itemize}


\item {\bf Trayectoria de una part\'icula cargada en el campo
  magn\'etico de la Tierra} (70 puntos)  

La Tierra tiene un campo magn\'etico que la protege de part\'iculas
cargadas. Por ejemplo, la aurora boreal es un fen\'omeno natural que se puede
entender a partir del moviemiento de part\'iculas cargadas que se
aceleran dentro de un campo magn\'etico y empizan a radiar energ\'ia
electromagn\'etica.  

El campo magn\'etico de la Tierra, cerca de ella y hasta una distancia
de 4$R_{T}$ donde $R_T=6378.1$km es el radio de  la Tierra, puede ser
aproximado por el campo de un dipolo 

\begin{equation}
\vec{B}_{dip}(\vec{r}) = \frac{\mu_0}{4\pi r^3}[3(\vec{M}\cdot\vec{
    r})\hat{r}-\vec{M}], 
\end{equation}

donde $\vec{r}=x\hat{x} + y\hat{y} + z\hat{z}$, $r=|\hat{r}|$ y
$\hat{r}=\vec{r}/r$. Para la Tierra tomamos ${\bf M}=-M\hat{z}$
  antiparalelo al eje $z$ porque el polo norte magn\'etico es cercano
  al polo sur geogr\'afico. 

En el ecuador magn\'etico $(x=1R_T,y=z=0)$ la intensidad del campo es
$B_0=3\times 10^{-5}$T. Substitutendo esto en la ecuaci\'on anterior
tenermos que $\mu M/4\pi=B_0R_T^3$. Con esto podemos escribir en
coordenadas cartesianas:

\begin{equation}
\vec{B}_{dip} = -\frac{B_0 R_T^3}{r^5}[3xz\hat{x} + 3yz\hat{y} + (2z^2
  - x^2 -y^2)\hat{z}].
\end{equation}

En este problema, las velocidades de las part\'iculas est\'an en el r\'egimen relativista. En este caso se tiene que la ecuaci\'on de la fuerza se escribe como

\begin{equation}
\frac{d(\gamma m \vec{v})}{dt} = q\vec{v}\times\vec{B}, 
\end{equation}
%
donde $\gamma=1/\sqrt{1-v^2/c^2}$. En este caso la energ\'ia cin\'etica de la part\'icula se escribe como $K=(\gamma -1)mc^2$, donde $m$ es la masa en reposo de la part\'icula. 

\begin{itemize}

\item (50 puntos) Escriba un programa en C que integre las ecuaciones
  de movimiento relativista de {\bf protones} que tienen una
  posici\'on inicial $(2R_e,0,0)$, energ\'ia cin\'etica inicial
  $E_{k}$ (medida en millones de electronvolts) y tienen una velocidad
  inicial descrita por $(0,v\sin\alpha, v\cos\alpha)$ donde $\alpha$
  es un \'angulo que se conoce como el pitch angle. El programa debe
  poder ejecutarse de la siguiente manera 
\begin{verbatim}
./particle_in_field.x kinetic_energy alpha
\end{verbatim}
donde \verb"kinetic_energy" es la energ\'ia cin\'etica inicial medida
en megaelectronvolts y \verb"alpha" es el \'angulo $\alpha$ medido en
grados. La trayectoria debe seguirse por $100$ segundos y al final
debe escribir el tiempo y las coordenadas $x,y,z$ de las posiciones en un archivo
llamado \verb"trayectoria_E_alpha.dat" donde \verb"E" es el valor de la
energ\'ia cin\'etica de la part\'icula y \verb"alpha" es el valor del
\'angulo $\alpha$.

\item (10 puntos) Escriba un programa en Python que grafique en el
  plano {\it xy}, y en {\it xyz} la trayectoria guardada en
  \verb"trayectoria_E_alpha.dat".

\item (10 puntos) Escriba un \verb"makefile" que enlace correctamente
  todos los pasos anteriores y cree las gr\'aficas necesarias.
\end{itemize}


\end{enumerate}


\end{document}
