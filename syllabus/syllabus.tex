% Física Computacional Syllabus
\documentclass[11pt]{article}
\usepackage[intlimits]{amsmath}
\usepackage{hyperref}
\usepackage{amsfonts}
\usepackage{amscd}
\usepackage{amssymb}
\usepackage{natbib}
\usepackage[spanish] {babel}
\textheight=25.5cm
\textwidth=16.0cm
\oddsidemargin=-0.5cm
\topmargin=-2.0cm
%\usepackage[latin1]{inputenc}
%\usepackage[utf8]{inputenc}



\begin{document}
\noindent
FISI 2028 M\'etodos Computationales
Semestre 2014 - 20\\
Martes y Mi\'ercoles 15:30 - 16:50 \\
Salon LL-204\\
Profesor: Jaime Forero, email: je.forero, Oficina Ip208\\
Monitores: \\
(Magistral \& Laboratorio) Nicol\'as Garavito, email: jn.garavito57\\
(Laboratorio) Luis Alberto Guti\'errez, email: la.gutierrez1280\\
(Laboratorio) Christian Poveda, email: cn.poveda542\\
(Magistral) Mar\'ia Camila Remolina, email: mc.remolina197\\


\section*{Objetivo}
El curso tiene como objetivo principal \emph{desarrollar en los
  estudiantes una adecuada actitud computacional, con la capacidad de
  discernir sobre los m\'etodos adecuados para solucionar cualquier
  problema y entender sus limitaciones.} 
 
En esta clase dar\'e enfasis a esa \emph{actitud computacional} que
corresponde al conjunto de habilidades para trabajar con computadores
en generar y procesar datos que correspondan a sistemas f\'isicos,
donde estos datos corresponden a una medici\'on o una simulaci\'on. 

\section*{Metodolog\'ia}
Esa \emph{actitud computacional} se desarrolla trabajando. Las
sesiones de f\'isica computacional ser\'an, sobre todo, una sesi\'on
de exploraci\'on, pr\'actica y experimentaci\'on. Para que esto
funcione es necesario que los estudiantes lleguen a clase despu\'es de
haber le\'ido sobre el tema correspondiente. 

El programa del curso tiene dos partes bien diferenciadas. La parte de
m\'etodos tradicionales de computo num\'erico y la parte de
\emph{carpinter\'ia} de software. La primera es probable que le sea
\'util a una fracci\'on de los asistentes al curso en su vida
profesional. La segunda parte le ser\'a \'util a \emph{todos}. 


\section*{Software}
\noindent Se usar\'an principalmente los notebooks de IPython
complementado con C. Tambi\'en se aceptan tareas en los siguientes
lenguajes de programaci\'on: FORTRAN 90/95, C++ y Python. No se
aceptar\'an tareas en Matlab, Mathematica o cualquier otro lenguaje
de programaci\'on que no este en la lista mencionada antes.  

\section*{Evaluaci\'on}

Hay 8 talleres para entregar. Los primeros 4 valen el 10\% cada uno,
los siguientes 4 valen el 15\% cada uno. No habr\'a parciales ni
examen final. Los primeros cuatro talleres ser\'an individuales. Los
\'ultimos cuatro talleres ser\'an en parejas.  Si en las entregas
individuales es claro que hubo trabajo en grupos entonces la nota de
todos los talleres individuales quedar\'a autom\'aticamente en cero
(0.0).  

Las entregas para los \'ultimos 4 talleres se har\'an en dos tiempos:
una primera entrega donde se muestre expl\'icitamente un borrador del
c\'odigo con comentarios, luego la entrega definitiva con el c\'odigo
completo. La primera entrega es una condici\'on necesaria para aceptar
la segunda. Solamente la segunda entrega recibe una nota. 

Esta materia se ve al mismo tiempo que el Laboratorio de M\'etodos
Computationales. El objetivo del Laboratorio es tener m\'as tiempo
para practicar todos lo visto en clase. De acuerdo a la
nota definiva en Laboratorio habr\'a un bono en la nota definitiva de
este curso. Siendo $x$ la nota de Laboratorio, el bono correspondiente
se calcula as\'i:
$4.0 < x \leq 4.4 \rightarrow 0.2$, $4.4< x\leq 4.8\rightarrow 0.3$, $4.8<x
\leq 5.0\rightarrow 0.5$.


El curso cuenta con un repositorio en github:
\url{https://github.com/forero/ComputationalMethods}. El material se
encuentra distribuido en las siguientes carpetas. 


\begin{itemize}
\item \texttt{hands\_on/}: ejemplos para hacer en clase.
\item \texttt{homework/}: enunciados y calificaciones de las tareas.
\item \texttt{notas/}: notas de clase.
\item \texttt{syllabus/}: programa del curso.
\end{itemize}
 
 

\section*{Programa}

\begin{center}
\begin{tabular}{|p{1.8cm}|p{6cm}|p{4.0cm}|p{4.5cm}|}
\hline
Semana & Teor\'ia & Carpinter\'ia & Taller \\\hline
1 (28.7) & Algoritmos	&Consola/Emacs/Gnuplot & \#1 (shell scripts)\\\hline
2 (4.8)& 	& Consola/Emacs - C  & \\\hline
3 (11.8)& 	&C  & \#2 (C)\\\hline
4 (18.8)& 	&Python &  \\\hline
5 (25.8)& 	&IPython Notebook (gr\'aficas) / Github Individual & \#3
(Python) \\\hline
6 (1.9)&      Soluci\'on de sistemas lineales de ecuaciones &  & \#4
(Matrices, Git, C)\\\hline
7 (8.9)&  Interpolaci\'on & Dise\~no de programas / Github Colaborando & \\ \hline
8 (15.9)& An\'alisis de Fourier - FFT &  Ipython / Numpy& \#5 (Fourier, Git, Python)\\\hline
9 (22.9) & {\bf Semana de trabajo individual} & &\\\hline 
10 (29.9) & Integraci\'on y derivaci\'on num\'erica & Ipython / Numpy & \\\hline
11 (6.10) & Ecuaciones diferenciales ordinarias (1er orden)& &  \#6
(EDO1, Derivadas, C, Python, Git)\\\hline
12 (13.10)& Ecuaciones diferenciales ordinarias (2do orden)&  Makefile &   \\\hline 
13 (20.10) & Ecuaciones diferenciales parciales & & \#7 (EDO2, EDP, C, Python, Git)\\\hline
14 (27.10) & M\'etodos Monte Carlo &   & \\\hline
15 (3.11) & MCMC para hacer fits &    & \#8 (MCMC, Python)\\\hline
16 (10.11) &  &    & \\\hline
\hline
\end{tabular}
\end{center}


\section*{Bibliograf\'ia}
\begin{itemize}
\item
\textit{A survey of Computational Physics}
. R. H. Landau, M. J. P\'aez, C. C. Bordeianu. Princeton Univ. Press. 2006
\item 
\textit{Statistical Mechanics: Algorithms and Computations.}
W. Krauth, Oxford Univ. Press. 
\item 
\textit{Introduction to Computational and Programming Using Python},
Guttag, J. V. The MIT Press. 2013.
\item 
\textit{The C programming language.}
 B. Kernighan \& D. Ritchie, Second Edition, Prentice Hall.
\item\url{http://software-carpentry.org/}
\item\url{http://xkcd.com/}
\end{itemize}

 

\end{document}
