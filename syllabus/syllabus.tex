% Física Computacional Syllabus
\documentclass[11pt]{article}
\usepackage[intlimits]{amsmath}
\usepackage{hyperref}
\usepackage{amsfonts}
\usepackage{amscd}
\usepackage{amssymb}
\usepackage{natbib}
\usepackage[spanish]{babel}
\textheight=25.5cm
\textwidth=16.0cm
\oddsidemargin=-0.5cm
\topmargin=-2.0cm
%\usepackage[latin1]{inputenc}
%\usepackage[utf8]{inputenc}



\begin{document}
\noindent
\textbf{FISI 2028 M\'etodos Computacionales}
Cod. 13834\\
Semestre 2015-I\\
Martes y Mi\'ercoles 15:30 - 16:50 \\
Sal\'on LL204\\
Profesor: Sebastian Perez Saaibi, email: \textbf{spsaaibi}\\
Monitores: \\
TBA \\
%(Magistral) Mar\'ia Camila Remolina, email: \textbf{mc.remolina197}\\
%(Laboratorio) Juan Nicol\'as Garavito, email:  \textbf{jn.garavito57}\\
%(Laboratorio) Luis Alberto Guti\'errez, email: la.gutierrez1280\\
%(Laboratorio) Christian Poveda, email: cn.poveda542\\
%\\


\section*{Objetivo}
El curso tiene como objetivo principal desarrollar en los
estudiantes una adecuada \emph{actitud computacional}, con la capacidad de
discernir sobre los m\'etodos y t\'ecnicas para solucionar cualquier 
problema y entender sus limitaciones.
 
El foco de la clase ser\'a esta \emph{actitud computacional} que
corresponde al conjunto de habilidades para trabajar con computadores
en generar y procesar datos para obtener intuici\'on a partir de ellos. Estos datos pueden
corresponder a mediciones o simulaciones sobre sistemas f\'isicos, biol\'ogicos, financieros o industriales, entre otros.

\section*{Metodolog\'ia}
Esa \emph{actitud computacional} se desarrolla trabajando. Las
sesiones de \textbf{M\'etodos computacionales} ser\'an, sobre todo 
enfocadas en la exploraci\'on, pr\'actica y experimentaci\'on. Para que esto
funcione es necesario que los estudiantes lleguen a clase despu\'es de
haber le\'ido sobre el tema correspondiente. 

El programa del curso tiene dos componentes diferenciados. La parte de
m\'etodos de computo num\'erico y la parte de
\emph{carpinter\'ia} de software. La parte de m\'etodos num\'ericos
ilustra como pasar de la matem\'atica a la computaci\'on num\'erica,
al igual que como se implementan algunos algoritmos en la pr\'actica. 
La parte de carpiter\'ia de software aumentar\'a la familiaridad con
la producci\'on cient\'ifica de software de los asistentes.


\section*{Software}
\noindent Se usar\'an principalmente: notebooks de IPython, Reportes en R Markdown (.Rmd) complementadas con C. Tambi\'en se aceptan tareas en los siguientes lenguajes de programaci\'on: FORTRAN 90/95, C++, Python, R. No se aceptar\'an tareas en Matlab, Mathematica o cualquier otro lenguaje
de programaci\'on que no este en la lista mencionada antes.  

\section*{Evaluaci\'on}

Hay 8 talleres para entregar. Los primeros 4 valen el 10\% cada uno,
los siguientes 4 valen el 15\% cada uno. \textbf{No habr\'a parciales ni
examen final}. Los primeros cuatro talleres ser\'an \textbf{individuales}. Los
\'ultimos cuatro talleres ser\'an en \textbf{parejas}.  Si en las entregas
individuales se detecta que hubo trabajo en grupos entonces la nota de
todos los talleres individuales quedar\'a autom\'aticamente en cero
\textbf{(0.0)}.  

Las entregas para los \'ultimos 4 talleres se har\'an en dos tiempos:
una primera entrega donde se muestre expl\'icitamente un borrador del
c\'odigo con comentarios, luego la entrega definitiva con el c\'odigo
completo. La primera entrega es una condici\'on necesaria para aceptar
la segunda. Solamente la segunda entrega recibe una nota. 

Esta materia se ve al mismo tiempo que el \emph{Laboratorio de M\'etodos
Computationales}. El objetivo del Laboratorio es tener m\'as tiempo
para practicar todos lo visto en clase. De acuerdo a la
nota definiva en Laboratorio habr\'a un bono en la nota definitiva de
este curso. Siendo $x$ la nota de Laboratorio, el bono correspondiente
se calcula as\'i:
$4.0 < x \leq 4.4 \rightarrow 0.2$, $4.4< x\leq 4.8\rightarrow 0.3$, $4.8<x
\leq 5.0\rightarrow 0.5$.


El curso cuenta con un repositorio en github:
\url{https://github.com/forero/ComputationalMethods}. El material se
encuentra distribuido en las siguientes carpetas. 


\begin{itemize}
\item \texttt{hands\_on/}: Ejemplos para hacer en clase.
\item \texttt{homework/}: Enunciados y calificaciones de las tareas.
\item \texttt{notas/}: Notas de clase.
\item \texttt{syllabus/}: Programa del curso.
\end{itemize}
 
 

\section*{Programa}

\begin{center}
\begin{tabular}{|p{2.1cm}|p{5.7cm}|p{4.0cm}|p{4.5cm}|}
\hline

Semana & Teor\'ia & Carpinter\'ia & Taller \\\hline
1 (1.20-21) & Algoritmos	& Consola/ SublimeText/ R (gr\'aficas) & \#1 (shell scripts)\\\hline
2 (1.27-28)& 	& Consola/SublimeText - C  & \\\hline
3 (2.3-4)& 	&C  & \#2 (C)\\\hline
4 (2.10-11)& 	&Python, R &  \\\hline
5 (2.17-18)& 	&IPython Notebook (gr\'aficas) / Github Individual / Rmd Reports & \#3
(Python) \\\hline
6 (2.24-25)&      Soluci\'on de sistemas lineales de ecuaciones &  & \#4
(Matrices, Git, C)\\\hline
7 (3.3-4)&  Interpolaci\'on & Dise\~no de programas / Github Colaborando & \\ \hline
8 (3.10-11)& An\'alisis de Fourier - FFT &  Ipython / Numpy& \#5 (Fourier, Git, Python)\\\hline
9 (3.17-18) & Integraci\'on y derivaci\'on num\'erica & Ipython / Numpy & \\\hline
10 (3.24-25) & Ecuaciones diferenciales ordinarias (1er orden)& &  \#6
(EDO1, Derivadas, Python, R, Git)\\\hline
11 (3.31-4.1) & {\bf Semana de trabajo individual} & &\\\hline
12 (4.7-8)& Ecuaciones diferenciales ordinarias (2do orden)&  Makefile &   \\\hline 
13 (4.14-15) & Ecuaciones diferenciales parciales & & \#7 (EDO2, EDP, Python, R, Git)\\\hline
14 (4.21-22) & M\'etodos Monte Carlo &   & \\\hline
15 (4.28-29) & MCMC para hacer fits &    & \#8 (MCMC, Python, R)\\\hline
16 (5.5-6) &  &    & \\\hline

\hline
\end{tabular}
\end{center}

\newpage

\section*{Referencias Bibliogr\'aficas}

\begin{itemize}
\item
\textit{Elements of Scientific Computing}
Tveito A., Langtangen H.P., Nielsen B.F., Cai X. Spinger. 2010.
\item
\textit{A survey of Computational Physics}
. R. H. Landau, M. J. P\'aez, C. C. Bordeianu. Princeton Univ. Press. 2006
\item 
\textit{Statistical Mechanics: Algorithms and Computations.}
W. Krauth, Oxford Univ. Press. 
\item 
\textit{Introduction to Computation and Programming Using Python},
Guttag, J. V. The MIT Press. 2013.
\item 
\textit{The C programming language.}
 B. Kernighan \& D. Ritchie, Second Edition, Prentice Hall.
\item\url{http://software-carpentry.org/}
\item\url{http://xkcd.com/}
\end{itemize}

 

\end{document}
