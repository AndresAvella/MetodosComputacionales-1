% Física Computacional Syllabus
\documentclass[12pt]{article}
\usepackage[intlimits]{amsmath}
\usepackage{amsfonts}
\usepackage{amscd}
\usepackage{amssymb}
\usepackage{natbib}
\usepackage[spanish] {babel}

%\usepackage[latin1]{inputenc}
%\usepackage[utf8]{inputenc}



\begin{document}
\noindent
FISI 2025 F\'isica Computacional \\
Semestre 2013 - 02\\
Martes y Jueves 1:00 - 2:20 \\
Salon LL-204\\
Profesor: Jaime Forero, email: je.forero\\
Monitor: Christian Poveda, email: cn.poveda542\\

\section*{Objetivo}
El curso tiene como objetivo principal \emph{desarrollar en los
  estudiantes una adecuada actitud computacional, con la capacidad de
  discernir sobre los m\'etodos adecuados para solucionar cualquier
  problema y entender sus limitaciones.} 
 
En esta clase dar\'e enfasis a esa \emph{actitud computacional} que
corresponde al conjunto de habilidades para trabajar con computadores
en generar y procesar datos que correspondan a sistemas f\'isicos,
donde estos datos corresponden a una medici\'on o una simulaci\'on. 

\section*{Metodolog\'ia}
Esa \emph{actitud computacional} se desarrolla trabajando. Las
sesiones de f\'isica computacional ser\'an, sobre todo, una sesi\'on
de exploraci\'on, pr\'actica y experimentaci\'on. Para que esto
funcione es necesario que los estudiantes lleguen a clase despu\'es de
haber le\'ido sobre el tema correspondiente. 

El programa del curso tiene dos partes bien diferenciadas. La parte de
m\'etodos tradicionales de computo num\'erico y la parte de
\emph{carpinter\'ia} de software. La primera es probable que le sea
\'util a una fracci\'on de los asistentes al curso en su vida
profesional. La segunda parte le ser\'a \'util a \emph{todos}. 

Las lecturas que se deben completar antes de clase tienen que ver con
la primera parte, sobre todo. Mientras que la segunda parte de
carpinter\'ia ser\'a trabajada en clase mientras se resuelven
problemas pr\'acticos de c\'omputo num\'erico. 

\section*{Software}
\noindent Se usar\'an principalmente los notebooks de IPython
complementado con C. Tambi\'en se aceptan tareas en los siguientes
lenguajes de programaci\'on: FORTRAN 90/95, C++ y Python. No se
aceptar\'an tareas en Matlab, Matem\'atica o cualquier otro lenguaje
de programaci\'on que no este en la lista mencionada antes.  

\section*{Evaluaci\'on}

Hay 8 talleres para entregar. Los primeros 4 valen el 10\% cada uno, los siguientes 4 valen el 15\% cada uno. No habr\'a parciales ni examen final. Los primeros cuatro talleres ser\'an individuales. Los \'ultimos cuatro talleres ser\'an en parejas. 

Las entregas para los \'ultimos 4 talleres se har\'an en dos tiempos: una primera entrega donde se muestre expl\'icitamente un borrador del c\'odigo con comentarios, luego la entrega definitiva con el c\'odigo completo. La primera entrega es una condici\'on necesaria para aceptar la segunda. Solamente esa segunda entrega recibe una nota.

 \newpage
\section*{Programa}

\begin{center}
\begin{tabular}{|p{1cm}|p{6cm}|p{5cm}|c|}
\hline
Sem. & Teor\'ia & Carpinter\'ia & Taller \\\hline
1 & Algoritmos	&Consola/Emacs & \#1\\
2 & 	& Consola/Emacs - C  & \#2\\
3 & 	&C  &    \\
4 & 	&Python & \#3\\
5 & 	&IPython Notebook (graficas) / Github Individual & \\
6 &      M\'inimos cuadrados y Principal Component Analysis & GSL & \\
7 &  Interpolaci\'on & Dise\~no de programas / Github Colaborando & \#4\\ 
8 &   & & \\
 & {\bf Semana de trabajo individual} & &\\
9 & An\'alisis de Fourier - FFT  (FFTW)&  Ipython & \#5 \\
10 & Integraci\'on y derivaci\'on num\'erica & Ipython &\\
11 & Ecuaciones diferenciales ordinarias (1er orden)& &\#6\\
12 & Ecuaciones diferenciales ordinarias (2do orden)&  Makefile & \ \\
13 & Ecuaciones diferenciales parciales &  & \#7\\
14 & M\'etodos Monte Carlo &   &  \\
15 & MCMC para hacer fits &    & \#8 \\

\hline
\end{tabular}
\end{center}


\section*{Bibliograf\'ia}
\begin{itemize}
\item
A survey of Computational Physics. R. H. Landau, M. J. P\'aez, C. C.
Bordeianu. Princeton Univ. Press. 2008 
\item
Statistical Mechanics: Algorithms and Computations. W. Krauth, Oxford Univ. Press. 
\item\verb"http://software-carpentry.org/"
\item\verb"https://github.com/forero/ComputationalPhysicsUniandes/"
\end{itemize}

 

\end{document}
