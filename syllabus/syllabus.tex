% Física Computacional Syllabus
\documentclass[12pt]{article}
\usepackage[intlimits]{amsmath}
\usepackage{amsfonts}
\usepackage{amscd}
\usepackage{amssymb}
\usepackage{natbib}
\usepackage[spanish] {babel}

%\usepackage[latin1]{inputenc}
%\usepackage[utf8]{inputenc}


\title{FISI 2025 F\'isica Computacional \\
Semestre 2013 - 01\\
Martes y Jueves 1:00 - 2:20 \\
Salon Z 122 - Lab. CompuFis
}
\author{Profesor: Jaime Forero}

\begin{document}
\maketitle
\section*{Objetivo}


\section*{Software}
\noindent Se usar\'an principalmente los notebooks de IPython. Tambi\'en se aceptan tareas en FORTRAN 90/95, C, C++ y Python.

\section*{Evaluaci\'on}
Habr\'an 10 talleres para entregar, cada uno con un valor del 9\%. Habr\'a  quizes sorpresa, que contar\'an por el 10\% restante. No habr\'a parciales ni examen final.

 
\section*{Programa}

\begin{center}
\begin{tabular}{|l|l|l|c|}
\hline
Sem. & & Taller &\\\hline
1 & 	&Consola/Emacs &\\
2 & 	&C  &*\\
3 & 	&Python / IPython Notebook &*\\
4 & Matrices y sistemas de ecuaciones lineales & Data &\\
5 & M\'inimos cuadrados & Data &\\
6 & Interpolaci\'on & Version Control (Git) &*\\
7 & Integraci\'on y derivaci\'on num\'erica & Version Control (Github)& *\\
8 & An\'alisis de Fourier - FFT  (FFTW)& Make& *\\
9 & Ecuaciones diferenciales Ordinarias & Make&\\
 & {\bf Semana de trabajo individual} & &\\
10 & Ecuaciones diferenciales ordinarias & Testing&\\
11 & Ecuaciones diferenciales ordinarias & Testing &*\\
12 & Ecuaciones diferenciales parciales & Dise\~no de programas&*\\
13 & M\'etodos Monte Carlo & Dise\~no de programas &*\\
14 & M\'etodo de diferencias finitas & C+Python &*\\
15 & M\'etodo de inferencia bayesiana& C+Python &*\\
\hline
\end{tabular}
\end{center}


\section*{Bibliograf\'ia}


\section*{P\'agina Web}
\begin{verbatim}
http://wwwprof.uniandes.edu.co/~je.forero/
https://github.com/forero/ComputationalPhysicsUniandes
\end{verbatim}

 

\end{document}
